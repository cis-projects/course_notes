%% Generated by Sphinx.
\def\sphinxdocclass{jupyterBook}
\documentclass[letterpaper,10pt,english]{jupyterBook}
\ifdefined\pdfpxdimen
   \let\sphinxpxdimen\pdfpxdimen\else\newdimen\sphinxpxdimen
\fi \sphinxpxdimen=.75bp\relax
\ifdefined\pdfimageresolution
    \pdfimageresolution= \numexpr \dimexpr1in\relax/\sphinxpxdimen\relax
\fi
%% let collapsible pdf bookmarks panel have high depth per default
\PassOptionsToPackage{bookmarksdepth=5}{hyperref}
%% turn off hyperref patch of \index as sphinx.xdy xindy module takes care of
%% suitable \hyperpage mark-up, working around hyperref-xindy incompatibility
\PassOptionsToPackage{hyperindex=false}{hyperref}
%% memoir class requires extra handling
\makeatletter\@ifclassloaded{memoir}
{\ifdefined\memhyperindexfalse\memhyperindexfalse\fi}{}\makeatother

\PassOptionsToPackage{warn}{textcomp}

\catcode`^^^^00a0\active\protected\def^^^^00a0{\leavevmode\nobreak\ }
\usepackage{cmap}
\usepackage{fontspec}
\defaultfontfeatures[\rmfamily,\sffamily,\ttfamily]{}
\usepackage{amsmath,amssymb,amstext}
\usepackage{polyglossia}
\setmainlanguage{english}



\setmainfont{FreeSerif}[
  Extension      = .otf,
  UprightFont    = *,
  ItalicFont     = *Italic,
  BoldFont       = *Bold,
  BoldItalicFont = *BoldItalic
]
\setsansfont{FreeSans}[
  Extension      = .otf,
  UprightFont    = *,
  ItalicFont     = *Oblique,
  BoldFont       = *Bold,
  BoldItalicFont = *BoldOblique,
]
\setmonofont{FreeMono}[
  Extension      = .otf,
  UprightFont    = *,
  ItalicFont     = *Oblique,
  BoldFont       = *Bold,
  BoldItalicFont = *BoldOblique,
]



\usepackage[Bjarne]{fncychap}
\usepackage[,numfigreset=1,mathnumfig]{sphinx}

\fvset{fontsize=\small}
\usepackage{geometry}


% Include hyperref last.
\usepackage{hyperref}
% Fix anchor placement for figures with captions.
\usepackage{hypcap}% it must be loaded after hyperref.
% Set up styles of URL: it should be placed after hyperref.
\urlstyle{same}

\addto\captionsenglish{\renewcommand{\contentsname}{Chapters}}

\usepackage{sphinxmessages}



        % Start of preamble defined in sphinx-jupyterbook-latex %
         \usepackage[Latin,Greek]{ucharclasses}
        \usepackage{unicode-math}
        % fixing title of the toc
        \addto\captionsenglish{\renewcommand{\contentsname}{Contents}}
        \hypersetup{
            pdfencoding=auto,
            psdextra
        }
        % End of preamble defined in sphinx-jupyterbook-latex %
        

\title{Project-Based Course Notes}
\date{Feb 13, 2022}
\release{}
\author{School of Computing and Information Systems, The University of Melbourne}
\newcommand{\sphinxlogo}{\vbox{}}
\renewcommand{\releasename}{}
\makeindex
\begin{document}

\pagestyle{empty}
\sphinxmaketitle
\pagestyle{plain}
\sphinxtableofcontents
\pagestyle{normal}
\phantomsection\label{\detokenize{chapter_0/introduction::doc}}


\sphinxAtStartPar
The purpose of these course notes is to act as a resource for students to consult throughout the semester while
undertaking one of the below subjects:
\begin{itemize}
\item {} 
\sphinxAtStartPar
\sphinxhref{https://handbook.unimelb.edu.au/2022/subjects/swen30006}{SWEN30006 Software Modelling and Design}

\item {} 
\sphinxAtStartPar
\sphinxhref{https://handbook.unimelb.edu.au/2022/subjects/swen90009}{SWEN90009 Software Requirements Analysis}

\item {} 
\sphinxAtStartPar
\sphinxhref{https://handbook.unimelb.edu.au/2022/subjects/swen90014}{SWEN90014 Masters Software Engineering Project}

\item {} 
\sphinxAtStartPar
\sphinxhref{https://handbook.unimelb.edu.au/2022/subjects/swen90013}{SWEN90013 Masters Advanced Software Project}

\item {} 
\sphinxAtStartPar
\sphinxhref{https://handbook.unimelb.edu.au/2022/subjects/comp90082}{COMP90082 Software Project}

\end{itemize}

\sphinxAtStartPar
These subjects all involve either the development of requirements of a software system or the implementation of that
system into code from requirements.

\sphinxAtStartPar
For many students, this is the first time you will be working in teams using agile methodologies, working with a
client, and potentially even the first time you are developing software requirements (rather than the requirements
being given to you from a university professor). As such, the learning curve for this subject can be steep for some.

\sphinxAtStartPar
This handbook should help you acclimate yourselves to content learned throughout the semester and can be used as a
resource to reference when you have questions.

\begin{sphinxadmonition}{attention}{Attention:}
\sphinxAtStartPar
This handbook will act as a living guide, so please feel free to raise an issue in the top right\sphinxhyphen{}hand side of this
website, if you want to raise suggestions or advice.

\sphinxAtStartPar
\sphinxincludegraphics{{github_issue}.png}
\end{sphinxadmonition}

\begin{DUlineblock}{0em}
\item[] \sphinxstylestrong{\Large Contributors}
\end{DUlineblock}

\sphinxAtStartPar
A big thank you to the contributors who worked on these course notes. You can get in contact with them
\sphinxhref{https://github.com/cis-projects/project\_based\_course\_notes}{here}.

\sphinxAtStartPar
This work was supported by the University of Melbourne Chancellery Academic – Dual Delivery Contributions grant.


\part{Chapters}


\chapter{Chapter 1: Agile Methodology}
\label{\detokenize{chapter_1/agile_methodology:chapter-1-agile-methodology}}\label{\detokenize{chapter_1/agile_methodology::doc}}
\sphinxAtStartPar
In this subject, you will form teams and operate according to the agile
methodology. This chapter will help familiarise students with the agile
framework for project management and software development.


\section{Software Development Lifecycles (SDLCs)}
\label{\detokenize{chapter_1/agile_methodology:software-development-lifecycles-sdlcs}}
\sphinxAtStartPar
Software development lifecycles (SDLCs) refer to a process for planning,
developing, testing, and deploying a software system. There are two main
types of SDLCs: formal and Agile.


\begin{savenotes}\sphinxattablestart
\centering
\begin{tabulary}{\linewidth}[t]{|T|T|}
\hline
\sphinxstyletheadfamily 
\sphinxAtStartPar
SDLC
&\sphinxstyletheadfamily 
\sphinxAtStartPar
Type
\\
\hline
\sphinxAtStartPar
\sphinxhref{https://www.workfront.com/project-management/methodologies/waterfall}{Waterfall}
&
\sphinxAtStartPar
Formal
\\
\hline
\sphinxAtStartPar
\sphinxhref{https://www.guru99.com/what-is-incremental-model-in-sdlc-advantages-disadvantages.html}{Incremental}
&
\sphinxAtStartPar
Formal
\\
\hline
\sphinxAtStartPar
\sphinxhref{https://www.tutorialspoint.com/sdlc/sdlc\_v\_model.htm\#:~:text=The\%20V\%2Dmodel\%20is\%20an,for\%20each\%20corresponding\%20development\%20stage.}{V\sphinxhyphen{}Model}
&
\sphinxAtStartPar
Formal
\\
\hline
\sphinxAtStartPar
\sphinxhref{https://www.atlassian.com/agile/kanban}{Kanban}
&
\sphinxAtStartPar
Agile
\\
\hline
\sphinxAtStartPar
{\hyperref[\detokenize{chapter_1/agile_methodology:scrum}]{\emph{Scrum}}}
&
\sphinxAtStartPar
Agile
\\
\hline
\sphinxAtStartPar
\sphinxhref{http://www.extremeprogramming.org/}{Extreme Programming}
&
\sphinxAtStartPar
Agile
\\
\hline
\end{tabulary}
\par
\sphinxattableend\end{savenotes}

\begin{sphinxadmonition}{attention}{Attention:}
\sphinxAtStartPar
You do not need to learn these SDLC methodologies \sphinxhyphen{} you will only make
use of Scrum. Links have been provided to the others in case you want to
learn more.
\end{sphinxadmonition}

\sphinxAtStartPar
Agile was formed after teams were frustrated by the rigidity of formal
methods, and their inability to adapt to change easily without exceeding
cost and time constraints. For example, in Waterfall, teams complete all
the requirements work before moving to design and subsequently
development. If the requirements changed during the development phase,
the team would need to start over, costing significant time and money,
something that is not feasible in many situations.


\section{Agile Manifesto}
\label{\detokenize{chapter_1/agile_methodology:agile-manifesto}}
\sphinxAtStartPar
Agile, much like its name, focuses on being able to adapt to change
rapidly through developing software incrementally. The agile manifesto
explains the core values and principles that are the basis of Agile:
\begin{itemize}
\item {} 
\sphinxAtStartPar
Individuals and interactions over processes and tools.

\item {} 
\sphinxAtStartPar
Working software over comprehensive documentation.

\item {} 
\sphinxAtStartPar
Customer collaboration over contract negotiation.

\item {} 
\sphinxAtStartPar
Responding to change over following a plan.

\end{itemize}


\section{Scrum Framework}
\label{\detokenize{chapter_1/agile_methodology:scrum-framework}}
\sphinxAtStartPar
For this subject, you will work in agile teams following the Scrum
framework.


\subsection{What is Scrum?}
\label{\detokenize{chapter_1/agile_methodology:what-is-scrum}}
\sphinxAtStartPar
Scrum is a framework used to implement an Agile mindset. It focuses on ensuring teams
work together, embodying the values from the agile manifesto. Scrum is based on teams
working iteratively, in time\sphinxhyphen{}boxed sprints (typically 2\sphinxhyphen{}4 weeks), with a predetermined
set of tasks. During sprint planning, the team decides which tasks to work on during
the upcoming sprint, to ensure they maximise value to their client. If requirements
change, the teams can easily adapt as sprints are short, and the subsequent sprints
planning can re\sphinxhyphen{}prioritise the changed requirements.


\subsection{Sprints}
\label{\detokenize{chapter_1/agile_methodology:sprints}}
\sphinxAtStartPar
Sprints are a short time\sphinxhyphen{}boxed period in which a scrum team endeavours to complete a set
amount of work. The duration of sprints can be determined by teams, but typically duration
is between 2\sphinxhyphen{}4 weeks. As sprints have several {\hyperref[\detokenize{chapter_1/agile_methodology:scrum-ceremonies}]{\emph{ceremonies}}} that must be
completed, teams should determine what works best for them.


\subsection{Scrum Roles}
\label{\detokenize{chapter_1/agile_methodology:scrum-roles}}
\sphinxAtStartPar
Scrum teams have three key roles: product owner, scrum master, and development team members


\subsubsection{Product Owner}
\label{\detokenize{chapter_1/agile_methodology:product-owner}}
\sphinxAtStartPar
The product owner ensures the team delivers the most value to the client. They have a strong
understanding of the project, which is used to prioritise user stories in upcoming sprints.

\sphinxAtStartPar
\sphinxincludegraphics{{product_owner}.png}

\sphinxAtStartPar
\sphinxstyleemphasis{Source:} \sphinxhref{https://www.atlassian.com/agile/scrum/roles}{Atlassian}

\sphinxAtStartPar
Product owners are required to maintain close ties with the client and
seek their validation and input. They are also the conduit for
communication with the client \sphinxhyphen{} all emails, meeting invites, etc. from
the team should be guided through the product owners to reach the
client.


\subsubsection{Scrum Master}
\label{\detokenize{chapter_1/agile_methodology:scrum-master}}
\sphinxAtStartPar
The scrum master is responsible for ensuring that the Scrum framework is
followed. Unlike a manager, the scrum master coaches the team rather
than leading them \sphinxhyphen{} scrum masters are \sphinxstylestrong{servant leaders}. They have no
authority to act as a manager in the traditional sense of being able to
allocate tasks. Instead, it is their responsibility to ensure that they
oversee how the project is tracking, and work to unblock developers to
continue meeting objectives.

\sphinxAtStartPar
\sphinxincludegraphics{{scrum_master}.png}

\sphinxAtStartPar
\sphinxstyleemphasis{Source:} \sphinxhref{https://www.atlassian.com/agile/scrum/roles}{Atlassian}


\subsubsection{Development Team}
\label{\detokenize{chapter_1/agile_methodology:development-team}}
\sphinxAtStartPar
The development team refers to the team members who are implementing the
system. In your project, \sphinxstylestrong{all team members} must be a part of the
development team (including the Scrum master and product owner).

\sphinxAtStartPar
\sphinxincludegraphics{{development_team}.png}

\sphinxAtStartPar
\sphinxstyleemphasis{Source:} \sphinxhref{https://www.atlassian.com/agile/scrum/roles}{Atlassian}

\sphinxAtStartPar
There are several other roles that teams may benefit from using. Please
note that these are not Scrum roles:
\begin{itemize}
\item {} 
\sphinxAtStartPar
Quality Assurance/Testing lead: Monitors testing other initiatives
to ensure the system is built to a high quality. Often, they are
responsible for creating a testing plan and ensuring the testing
objectives are met

\item {} 
\sphinxAtStartPar
Frontend lead: Oversees the frontend development of the project and
is typically responsible for setting up and configuring the frontend
codebase. Additionally, plays a key role in any decision\sphinxhyphen{}making
about frontend architecture.

\item {} 
\sphinxAtStartPar
Backend lead: Oversees the backend development of the project and is
typically responsible for setting up and configuring the backend
codebase. Additionally, plays a key role in any decision\sphinxhyphen{}making
about backend architecture.

\end{itemize}


\subsection{Scrum Artefacts}
\label{\detokenize{chapter_1/agile_methodology:scrum-artefacts}}
\sphinxAtStartPar
Managing work in a scrum team involves the use of two artefacts:
\begin{itemize}
\item {} 
\sphinxAtStartPar
Product backlog: Primary list of work that needs to be done which is
maintained by the product owner.

\item {} 
\sphinxAtStartPar
Sprint backlog: The list of work that needs to be completed in the
current sprint.

\end{itemize}


\subsection{Scrum Ceremonies}
\label{\detokenize{chapter_1/agile_methodology:scrum-ceremonies}}
\sphinxAtStartPar
Scrum ceremonies refer to a set of meetings that are used to manage the
development of a project. These ceremonies are important in facilitating
team communication and reducing the feedback loop. The table below shows
the key details of the Scrum ceremonies.

\sphinxAtStartPar
\sphinxincludegraphics{{scrum_ceremonies}.svg}

\sphinxAtStartPar
\sphinxstyleemphasis{Source:} \sphinxhref{https://www.atlassian.com/agile/scrum/sprints}{Atlassian}


\begin{savenotes}\sphinxattablestart
\centering
\begin{tabulary}{\linewidth}[t]{|T|T|T|T|T|T|T|}
\hline
\sphinxstyletheadfamily 
\sphinxAtStartPar
Ceremony
&\sphinxstyletheadfamily 
\sphinxAtStartPar
When
&\sphinxstyletheadfamily 
\sphinxAtStartPar
Who
&\sphinxstyletheadfamily 
\sphinxAtStartPar
Preparation
&\sphinxstyletheadfamily 
\sphinxAtStartPar
Duration
&\sphinxstyletheadfamily 
\sphinxAtStartPar
Process
&\sphinxstyletheadfamily 
\sphinxAtStartPar
Outcome
\\
\hline
\sphinxAtStartPar
Sprint planning
&
\sphinxAtStartPar
Start of each sprint
&
\sphinxAtStartPar
Development team
&
\sphinxAtStartPar
Product owner should have a prioritised product backlog
&
\sphinxAtStartPar
1 hour per week of sprint. E.g., 2\sphinxhyphen{}week sprints have a 2\sphinxhyphen{}hour session
&
\sphinxAtStartPar
1. Product owner presents product backlog2. Discussions with team about high priority features, which are then broken down into smaller tasks3. Estimate the effort required for tasks.4. Team agrees on the tasks set out for the sprint, and tasks are moved to the sprint backlog.
&
\sphinxAtStartPar
Populated sprint backlog in the team’s task tracking tool.
\\
\hline
\sphinxAtStartPar
Stand\sphinxhyphen{}up
&
\sphinxAtStartPar
Weekly
&
\sphinxAtStartPar
Development team and supervisor
&
\sphinxAtStartPar
None
&
\sphinxAtStartPar
<15 minutes
&
\sphinxAtStartPar
Each team member says what they have done, what they will be working on and any blockers.
&
\sphinxAtStartPar

\\
\hline
\sphinxAtStartPar
Sprint review
&
\sphinxAtStartPar
End of each sprint
&
\sphinxAtStartPar
Development team and client
&
\sphinxAtStartPar
None
&
\sphinxAtStartPar
1 hour
&
\sphinxAtStartPar
Product owner presents the work completed in previous sprint and seeks feedback.
&
\sphinxAtStartPar
Populated sprint review page in the team’s document repository.
\\
\hline
\sphinxAtStartPar
Sprint retrospective
&
\sphinxAtStartPar
End of each sprint
&
\sphinxAtStartPar
Development team
&
\sphinxAtStartPar
None
&
\sphinxAtStartPar
20 mins
&
\sphinxAtStartPar
Scrum master facilitates session to establish what the team thought went well, what didn’t work and what actions the team could do to improve.
&
\sphinxAtStartPar
Populated retrospective page in the team’s document repository.
\\
\hline
\end{tabulary}
\par
\sphinxattableend\end{savenotes}

\sphinxAtStartPar
We have outlined how the scrum framework can assist teams in following
the agile manifesto. Now we will discuss how development requirements
can be represented.


\section{Representing Requirements the Agile Way}
\label{\detokenize{chapter_1/agile_methodology:representing-requirements-the-agile-way}}
\sphinxAtStartPar
Product requirements are a way of defining a product’s purpose,
features, functionality, and behaviour. It serves as a common place to
develop and guide understanding for the technical team (the developers)
and the client to help build the product.

\sphinxAtStartPar
These requirements are then represented as initiatives, epics, tasks,
and subtasks.

\sphinxAtStartPar
\sphinxincludegraphics{{requirements}.png}

\sphinxAtStartPar
\sphinxstyleemphasis{Source:} \sphinxhref{https://www.atlassian.com/agile/project-management/user-stories}{Atlassian}


\subsection{Initiatives}
\label{\detokenize{chapter_1/agile_methodology:initiatives}}
\sphinxAtStartPar
In this subject you will only have one concurrent project, so the
project will count as the initiative. In a larger business where there
might be 2 or more applications under development simultaneously, there
would be 2 or more initiatives to cover each development.


\subsection{Epics}
\label{\detokenize{chapter_1/agile_methodology:epics}}
\sphinxAtStartPar
An epic refers to a large body of work that can be broken down into user
stories. Epics cluster user stories that have a similar higher\sphinxhyphen{}level
objective.

\sphinxAtStartPar
Epics are a useful way to organise and create a hierarchy out of the
work. The goal of breaking down epics into user stories is to reduce the
amount of work required for the project into small, incremental tasks so
that value is delivered throughout the life of the project through the
completion of these smaller tasks. For client\sphinxhyphen{}based projects, this means
that you can deliver incremental progress and demonstrate value much
easier than if the tasks remained large pieces of work.

\sphinxAtStartPar
Epics generally have a duration greater than one sprint. As the team
progresses through the project and develops a better understanding of
the project and its requirements, user stories will naturally be added
and removed from an epic. This is exactly what we want in an epic
following agile methodology: a flexible scope that changes based on
client feedback and team progress.


\subsection{User Stories}
\label{\detokenize{chapter_1/agile_methodology:user-stories}}
\sphinxAtStartPar
A user story is the smallest unit of work in the agile framework. A user
story may have one or many subtasks, but user stories are considered the
smallest piece of work that constitutes an end goal from a user’s
perspective. A user story is an end goal, not a feature, expressed from
the perspective of the software user.

\sphinxAtStartPar
As discussed, user stories refer to a way of writing requirements from
the perspective of the end\sphinxhyphen{}user. By doing this, requirements follow the
agile methodology and are focused on the value delivered to the
end\sphinxhyphen{}user.

\sphinxAtStartPar
A key component of agile software development is putting people first,
and a user story puts end\sphinxhyphen{}users at the centre of the conversation. User
stories use non\sphinxhyphen{}technical language to provide context for the
development team. From reading a user story, the developer doing the
work required for that user story knows what they are building, why they
are building, for whom they are building it, and what value it creates.


\subsubsection{How to Write a User Story}
\label{\detokenize{chapter_1/agile_methodology:how-to-write-a-user-story}}
\sphinxAtStartPar
When first created, user stories consist of only one sentence in simple
language meant to outline the desired outcome \sphinxhyphen{} they don’t go into
detail. User stories are expanded upon later and more details are added
to the user story, as agreed by the team, as work begins on the issue.

\sphinxAtStartPar
User stories are written to be non\sphinxhyphen{}technical following the format:

\sphinxAtStartPar
As a {[}role{]}, I want to {[}goal{]}, so that I can {[}benefit{]}.

\sphinxAtStartPar
As an example, a company like Facebook might create a user story for new
functionality for their website with the following format:\sphinxstyleemphasis{As an individual user, I want to upload a photo to my profile picture,
so that people who search for my name can see a photo of me.}


\subsubsection{Story Points: User Story Estimation Metric}
\label{\detokenize{chapter_1/agile_methodology:story-points-user-story-estimation-metric}}
\sphinxAtStartPar
As user stories reflect requirements that a developer must implement,
they should have a corresponding estimate for the effort required to
develop the functionality. This is necessary for time\sphinxhyphen{}boxed projects to
estimate the time to delivery for new functionality. However,
estimations in the Agile world are not based on time, but instead use
\sphinxstylestrong{story points}.

\sphinxAtStartPar
Story points estimate a task by considering the amount of work that
needs to be done, the difficulty of the task and any potential risk or
uncertainty involved. This is a better estimate as using the time taken
to complete a task can vary from developer to developer depending on
their level of experience.

\sphinxAtStartPar
Various scales can be used for story points. Two common methods used
are:
\begin{itemize}
\item {} 
\sphinxAtStartPar
Fibonacci sequence (1, 2, 3, 5, 8…)

\item {} 
\sphinxAtStartPar
T\sphinxhyphen{}shirt sizing (XS, S, M, L, XL)

\end{itemize}

\sphinxAtStartPar
Regardless of which estimation technique your team chooses to use, you
should have the estimation technique documented in your team’s document
repository.


\subsubsection{Estimation Process}
\label{\detokenize{chapter_1/agile_methodology:estimation-process}}
\sphinxAtStartPar
The estimation process occurs during {\hyperref[\detokenize{chapter_1/agile_methodology:scrum-ceremonies}]{\emph{sprint
planning}}}. There are several ways to estimate the
story points of a user story \sphinxhyphen{} we will look at planning poker today:
\begin{enumerate}
\sphinxsetlistlabels{\arabic}{enumi}{enumii}{}{.}%
\item {} 
\sphinxAtStartPar
All members are given cards that have a story point value. If using
Fibonacci sequence, would have cards for 0, 1, 2, 3, 5, 8, 13…

\item {} 
\sphinxAtStartPar
Product owner reads out a user story and clarifies details if anyone
has any questions.

\item {} 
\sphinxAtStartPar
Team discusses how they will handle the task and what skills are
required to understand the approach.

\item {} 
\sphinxAtStartPar
Each member picks a card with a story point value they feel matches
the user story just discussed, and places it face\sphinxhyphen{}down.

\item {} 
\sphinxAtStartPar
Once all team members have chosen a card, the team turns over all
the cards and discusses.

\item {} 
\sphinxAtStartPar
Once the team reaches a consensus on the user story’s estimate, the
product owner notes down the value.

\item {} 
\sphinxAtStartPar
Repeat until all user stories product owner has prepared are
complete.

\end{enumerate}

\sphinxAtStartPar
The estimates should be added to your chosen task tracking tool.

\begin{sphinxadmonition}{note}{Extra Resources}


\begin{savenotes}\sphinxattablestart
\centering
\begin{tabulary}{\linewidth}[t]{|T|T|}
\hline
\sphinxstyletheadfamily 
\sphinxAtStartPar
Link
&\sphinxstyletheadfamily 
\sphinxAtStartPar
Description
\\
\hline
\sphinxAtStartPar
\sphinxhref{https://www.planningpoker.com/}{Planning Poker}
&
\sphinxAtStartPar
A free tool that gamifies the process.
\\
\hline
\end{tabulary}
\par
\sphinxattableend\end{savenotes}
\end{sphinxadmonition}


\subsubsection{User Story Prioritisation}
\label{\detokenize{chapter_1/agile_methodology:user-story-prioritisation}}
\sphinxAtStartPar
To deliver greater value to your client, you will want to invest your
efforts in tasks that are of importance first. The importance of tasks
is decided in discussions with the client (it \sphinxstylestrong{is not} up to the
development team to decide priority). One method of displaying the
priority of a task is the MoSCoW task prioritisation method. Priorities
are broken down into different levels of priority:
\begin{itemize}
\item {} 
\sphinxAtStartPar
\sphinxstyleemphasis{Must have:} features that must be delivered or the software will not
create the expected value for the client;

\item {} 
\sphinxAtStartPar
\sphinxstyleemphasis{Should have:} features that have significant value to the client and
should be delivered, but not considered crucial;

\item {} 
\sphinxAtStartPar
\sphinxstyleemphasis{Could have:} features that the client considers nice to have but will
not have a material impact to value, if not delivered; and

\item {} 
\sphinxAtStartPar
\sphinxstyleemphasis{Won’t have:} out\sphinxhyphen{}of\sphinxhyphen{}scope features; useful as next steps for your
project as potential improvements for future releases.

\end{itemize}

\begin{sphinxadmonition}{note}{Extra Resources}


\begin{savenotes}\sphinxattablestart
\centering
\begin{tabulary}{\linewidth}[t]{|T|T|}
\hline
\sphinxstyletheadfamily 
\sphinxAtStartPar
Link
&\sphinxstyletheadfamily 
\sphinxAtStartPar
Description
\\
\hline
\sphinxAtStartPar
\sphinxhref{https://en.wikipedia.org/wiki/MoSCoW\_method}{MoSCoW method}
&
\sphinxAtStartPar
Further discussion of the MoSCoW method.
\\
\hline
\sphinxAtStartPar
\sphinxhref{https://www.atlassian.com/agile/scrum}{Scrum Guide}
&
\sphinxAtStartPar
Comprehensive Scrum guide.
\\
\hline
\sphinxAtStartPar
\sphinxhref{https://www.mountaingoatsoftware.com/agile/scrum/meetings/sprint-planning-meeting}{Sprint planning meeting}
&
\sphinxAtStartPar
Details on running your sprint planning meeting.
\\
\hline
\end{tabulary}
\par
\sphinxattableend\end{savenotes}
\end{sphinxadmonition}

\sphinxAtStartPar
Once written, these stories should be documented centrally in the team’s
document repository:

\sphinxAtStartPar
\sphinxincludegraphics{{confluence}.png}


\subsubsection{User Story Mapping}
\label{\detokenize{chapter_1/agile_methodology:user-story-mapping}}
\sphinxAtStartPar
User story maps are intended to spark collaboration across agile team members, while providing them
with the bigger picture of how the backlog stories fit together into a larger vision of the product your team
is building. In agile team, product work exists as discrete backlog tasks and so visual tools are extremely helpful
in communicating deadlines, goals, and the final product to developers in the team, as well as the client.
This guarantees everyone is working towards the same ultimate goal.

\sphinxAtStartPar
Once created, the Product Owner is responsible for driving the creationg of the user story map. However, it is
a group exercise and should be done by the entire team. Once created, it should then be communicated to the client.


\paragraph{When to Create User Story Maps}
\label{\detokenize{chapter_1/agile_methodology:when-to-create-user-story-maps}}
\sphinxAtStartPar
The user story map can be created before or after user story prioritisation. Sometimes it can help to visualise
user stories when prioritising them.

\begin{sphinxadmonition}{note}{Extra Resources}

\sphinxAtStartPar
To aid you in creating a user story map, you can make use of templates.


\begin{savenotes}\sphinxattablestart
\centering
\begin{tabulary}{\linewidth}[t]{|T|}
\hline
\sphinxstyletheadfamily 
\sphinxAtStartPar
Link
\\
\hline
\sphinxAtStartPar
\sphinxhref{https://miro.com/templates/user-story-map/}{Miro template}
\\
\hline
\sphinxAtStartPar
\sphinxhref{https://www.lucidchart.com/blog/how-to-create-a-user-story-map}{LucidChart template}
\\
\hline
\end{tabulary}
\par
\sphinxattableend\end{savenotes}
\end{sphinxadmonition}

\sphinxAtStartPar
This is an example of a user story map:

\sphinxAtStartPar
\sphinxincludegraphics{{user_story_map}}


\chapter{Chapter 2: Client Communications}
\label{\detokenize{chapter_2/client_communications:chapter-2-client-communications}}\label{\detokenize{chapter_2/client_communications::doc}}
\sphinxAtStartPar
In this subject, your team will be delivering a project to a client.
Most students are likely to only have had experience mostly with
university assignments and will be unfamiliar with how to interaction
with a client.

\sphinxAtStartPar
Your client will be assigned by the Subject Coordinator, and once your
client is assigned, you should contact them and work closely with them
throughout the semester.

\sphinxAtStartPar
You will need them to:
\begin{enumerate}
\sphinxsetlistlabels{\arabic}{enumi}{enumii}{}{.}%
\item {} 
\sphinxAtStartPar
Aid your team in developing requirements for your system.

\item {} 
\sphinxAtStartPar
Review progress after each sprint.

\item {} 
\sphinxAtStartPar
Provide feedback for prototypes.

\end{enumerate}

\sphinxAtStartPar
It is imperative that you communicate all important product decisions
with your client and involve them by seeking their feedback.

\sphinxAtStartPar
Communication with the client takes place through the Product Owner \sphinxhyphen{}
they are the conduit of information between the team and the client, and
as such, will be responsible for emailing the client with updates,
invitations for meetings, etc.


\section{Making Initial Contact with Your Client}
\label{\detokenize{chapter_2/client_communications:making-initial-contact-with-your-client}}
\sphinxAtStartPar
You should contact your client via their provided email address. Your
initial contact should introduce your team members and decide on a
convenient time to conduct a first meeting.

\sphinxAtStartPar
To decide a common time to meet, you can use a scheduling tool, like
\sphinxhref{https://www.when2meet.com}{When2Meet}, to elicit their availability.


\subsection{Creating Meeting Invites}
\label{\detokenize{chapter_2/client_communications:creating-meeting-invites}}
\sphinxAtStartPar
Once an available time has been decided with your client, the Product
Owner should send them a calendar invite for the meeting. This way, both
your team and the client will not forget. Instructions on how to do this
can be found \sphinxhref{https://support.google.com/calendar/answer/72143?hl=en\&co=GENIE.Platform\%3DDesktop\#zippy=}{here}.

\sphinxAtStartPar
If the meeting will take place virtually, you can make use of the free
Zoom account provided by the university to schedule Zoom meetings.
Please follow the instructions to set up Zoom scheduling with your
university Gmail account \sphinxhref{https://workspace.google.com/marketplace/app/zoom\_for\_gmail/585972765488}{here}.


\section{During Client Meetings}
\label{\detokenize{chapter_2/client_communications:during-client-meetings}}
\sphinxAtStartPar
Once a meeting is set with the client, the team should set up the
Confluence page for the meeting \sphinxhyphen{} please see the chapter on document
repositories for more information.

\sphinxAtStartPar
During the meeting, there are several roles that need to be filled by
members of the development team:


\begin{savenotes}\sphinxattablestart
\centering
\begin{tabulary}{\linewidth}[t]{|T|T|}
\hline
\sphinxstyletheadfamily 
\sphinxAtStartPar
Roles
&\sphinxstyletheadfamily 
\sphinxAtStartPar
Responsibilities
\\
\hline
\sphinxAtStartPar
Meeting Coordinator
&
\sphinxAtStartPar
It is generally a good idea to set one person as the meeting coordinator. This person is not responsible for speaking the entire time, but they are responsible for making sure the meeting has structure and direction, and that all questions in the agenda are asked and answered. All team members are encouraged to talk and ask questions during a meeting.
\\
\hline
\sphinxAtStartPar
Minute Taker
&
\sphinxAtStartPar
This person is responsible for ensuring all discussions are well\sphinxhyphen{}documented (in Confluence). They are also responsible for emailing the minutes taken during a meeting to the client (within a day or two of the meeting).
\\
\hline
\end{tabulary}
\par
\sphinxattableend\end{savenotes}


\section{After Client Meetings}
\label{\detokenize{chapter_2/client_communications:after-client-meetings}}
\sphinxAtStartPar
At the conclusion of a client meeting, the Minute Taker should email the
client with the minutes they wrote. You don’t need to send the entire
copy of notes (which can take a long time for your client to read). You
could write down the main points in the meeting notes. Doing so will
allow the client to correct and mistakes and makes sure the team does
not waste time on work that is incorrect.


\section{Sharing a Team Calendar}
\label{\detokenize{chapter_2/client_communications:sharing-a-team-calendar}}
\sphinxAtStartPar
In large teams, it can be difficult to keep up with the number of
meetings that take place. There are tools that take out the hassle of
coordinating meeting invitations and reminders.

\sphinxAtStartPar
One that works well is \sphinxhref{https://slack.com/intl/en-au/help/articles/206329808-Google-Calendar-for-Slack}{Google Calendar for Slack}.
It can remind team members of meetings from directly in Slack if this is
your chosen communications channel. If you have chosen a different
application, you are welcome to search for plug\sphinxhyphen{}ins that work between it
and Google Calendar (the default for university).


\section{Logging Client Communications}
\label{\detokenize{chapter_2/client_communications:logging-client-communications}}
\sphinxAtStartPar
All important decisions and milestones reached with the client via email
should be logged in the team’s Confluence. This is to foster a shared
understanding of agreements reached with the client. Your supervisor
generally will not attend client meetings, so this also allows them to
stay informed. You can do this by creating a client log in your document
repository:

\sphinxAtStartPar
\sphinxincludegraphics{{confluence1}.png}

\sphinxAtStartPar
The communication itself should then be logged:

\sphinxAtStartPar
\sphinxincludegraphics{{communications}.png}

\begin{sphinxadmonition}{note}{Extra Resources}

\sphinxAtStartPar
One of our tutors made a helpful video to show how to organise meetings.
\end{sphinxadmonition}

\begin{sphinxuseclass}{cell}\begin{sphinxVerbatimInput}

\begin{sphinxuseclass}{cell_input}
\begin{sphinxVerbatim}[commandchars=\\\{\}]
\PYG{k+kn}{from} \PYG{n+nn}{IPython}\PYG{n+nn}{.}\PYG{n+nn}{display} \PYG{k+kn}{import} \PYG{n}{YouTubeVideo}

\PYG{n}{YouTubeVideo}\PYG{p}{(}\PYG{l+s+s1}{\PYGZsq{}}\PYG{l+s+s1}{VKmzozlgc7Y}\PYG{l+s+s1}{\PYGZsq{}}\PYG{p}{)}
\end{sphinxVerbatim}

\end{sphinxuseclass}\end{sphinxVerbatimInput}
\begin{sphinxVerbatimOutput}

\begin{sphinxuseclass}{cell_output}
\noindent\sphinxincludegraphics{{_build/jupyter_execute/chapter_0/client_communications_1_0}.jpg}

\end{sphinxuseclass}\end{sphinxVerbatimOutput}

\end{sphinxuseclass}

\chapter{Chapter 3: Task Tracking}
\label{\detokenize{chapter_3/task_tracking:chapter-3-task-tracking}}\label{\detokenize{chapter_3/task_tracking::doc}}
\sphinxAtStartPar
Task tracking, also known as task management, is far more than a simple
to\sphinxhyphen{}do list. It means tracking tasks from beginning to end, delegating
subtasks to teammates, and setting deadlines to make sure projects are
done on time.

\sphinxAtStartPar
There are several benefits to using task tracking software:
\begin{enumerate}
\sphinxsetlistlabels{\arabic}{enumi}{enumii}{}{.}%
\item {} 
\sphinxAtStartPar
Centralise tasks, increasing team visibility of task progress.

\item {} 
\sphinxAtStartPar
Prioritise tasks as a team.

\item {} 
\sphinxAtStartPar
Improve collaboration, allowing any team member to work across any
task.

\item {} 
\sphinxAtStartPar
Track team progress.

\end{enumerate}

\sphinxAtStartPar
As discussed in agile methodology, task tracking is an important part of
representing requirements in an agile way.

\sphinxAtStartPar
In this subject, you are required to make use of a task tracking tool.
As a team, you will need to agree on a tool to use. There are a several
tools to choose from (this list is not exhaustive):


\begin{savenotes}\sphinxattablestart
\centering
\begin{tabulary}{\linewidth}[t]{|T|T|}
\hline
\sphinxstyletheadfamily 
\sphinxAtStartPar
Name
&\sphinxstyletheadfamily 
\sphinxAtStartPar
Description
\\
\hline
\sphinxAtStartPar
\sphinxhref{https://www.atlassian.com/software/jira}{Jira}
&
\sphinxAtStartPar
The industry\sphinxhyphen{}leading task tracking software. A good tool to learn if you want to enter the software industry.
\\
\hline
\sphinxAtStartPar
\sphinxhref{https://trello.com/en}{Trello}
&
\sphinxAtStartPar
A very popular tool. Less feature\sphinxhyphen{}rich than Jira but quicker and easier to set up.
\\
\hline
\sphinxAtStartPar
\sphinxhref{https://docs.github.com/en/issues/organizing-your-work-with-project-boards/managing-project-boards/about-project-boards}{GitHub Project Board}
&
\sphinxAtStartPar
New, compared to the other tools. Easy to set up as it comes with your repository but lacking many features.
\\
\hline
\sphinxAtStartPar
\sphinxhref{https://asana.com/?noredirect}{asana}
&
\sphinxAtStartPar
Like Trello, but a more mobile\sphinxhyphen{}friendly application.
\\
\hline
\sphinxAtStartPar
\sphinxhref{https://linear.app/}{Linear}
&
\sphinxAtStartPar
Much like asana.
\\
\hline
\end{tabulary}
\par
\sphinxattableend\end{savenotes}


\section{Setting Up Your Team’s Task Management}
\label{\detokenize{chapter_3/task_tracking:setting-up-your-team-s-task-management}}
\sphinxAtStartPar
Once your team has decided on a tool to use, the next step is to
populate it.

\sphinxAtStartPar
Regardless of your choice of task tracking tools, all require similar
set\sphinxhyphen{}ups and the same level of detail.


\subsection{Requirements}
\label{\detokenize{chapter_3/task_tracking:requirements}}
\sphinxAtStartPar
As discussed in agile methodology, all requirements of the system must
be translated into epics, user stories, subtasks, etc. in your chosen
task tracking tool.

\begin{sphinxadmonition}{note}{Note:}
\sphinxAtStartPar
Please see the chapter on requirements elicitation to understand how to elicit requirements from a client.
\end{sphinxadmonition}


\subsection{Backlog}
\label{\detokenize{chapter_3/task_tracking:backlog}}
\sphinxAtStartPar
The backlog contains all pieces of work for the semester. Any work
completed by the development team should have an associated task in the
tracking tool. At the beginning of the semester, the Product Owner
should populate the backlog with all requirements received from the
initial client meeting.

\sphinxAtStartPar
The backlog will look like this once populated:

\sphinxAtStartPar
\sphinxincludegraphics{{backlog}.png}

\sphinxAtStartPar
Once the backlog is populated, the Product Owner and client should
jointly determine the priority of each task. The priorities of tasks
help the development team determine what task to pick up next. It is
expected the development team will work on high priority tasks first
before moving on to lower priority tasks.

\sphinxAtStartPar
By default, all tasks are created with medium priority:

\sphinxAtStartPar
\sphinxincludegraphics{{priority_1}.png}

\sphinxAtStartPar
However, the priority can be changed to any of the below by editing the
task:

\sphinxAtStartPar
\sphinxincludegraphics{{priority_2}.png}

\sphinxAtStartPar
The priority of all user stories in the backlog should reflect the
prioritisation given by the client (please see the chapter on
requirements elicitation).

\sphinxAtStartPar
Over the course of the year, the development team then uses the backlog
as its single source of truth for work, and as the project progresses,
work should be added and removed from the backlog.


\subsection{Workflow}
\label{\detokenize{chapter_3/task_tracking:workflow}}
\sphinxAtStartPar
As a team, you should have a defined flow that all work must follow. A
workflow covers all steps that must be done for work to be considered
“ready for development” and then subsequently “ready for review” and
then “complete”. Creating a workflow ensures that all team members
understand and follow the same process, thus ensuring a high quality of
work completed with far less oversight necessary. As workload increases,
having a defined workflow ensures that the team’s processes are
scalable.


\subsubsection{Defining a Workflow}
\label{\detokenize{chapter_3/task_tracking:defining-a-workflow}}
\sphinxAtStartPar
When defining a workflow, it can be tempting to make it complicated \sphinxhyphen{}
avoid this temptation. Most important when creating a workflow is that
it is understood by the team and highly repeatable.
\begin{enumerate}
\sphinxsetlistlabels{\arabic}{enumi}{enumii}{}{.}%
\item {} 
\sphinxAtStartPar
All tasks that are created are first created in the backlog. Tasks
that are selected for development are then moved into the upcoming
sprint.

\item {} 
\sphinxAtStartPar
Once a developer commences work on an issue, the issue is moved to
the next status.

\item {} 
\sphinxAtStartPar
Once development work has finished, the task should be moved to the
next status for review by teammate(s). The task must be reviewed by
another team member (not the one who completed the work).

\item {} 
\sphinxAtStartPar
If the review passes successfully (no bugs are found), the task
should be moved to a completed status. If it fails review, it should
be moved backwards to the prior status so the developer can address
the bugs found.

\end{enumerate}


\subsubsection{An Example Workflow}
\label{\detokenize{chapter_3/task_tracking:an-example-workflow}}
\sphinxAtStartPar
The following is just one example of a workflow, and you are welcome to
define your own that best suits your team \sphinxhyphen{} this is by no means the best
workflow.

\sphinxAtStartPar
\sphinxincludegraphics{{workflow1}.png}

\sphinxAtStartPar
\sphinxstyleemphasis{Source:} \sphinxhref{https://www.atlassian.com/agile/project-management/workflow}{Atlassian}

\sphinxAtStartPar
Status: Open

\sphinxAtStartPar
This is the status that is assigned to all newly created tasks in the
current sprint. For a task to be able to be moved from status open to in
progress, it must have a definition of ready (DoR). A task with a
definition of ready means it can be picked up by any developer and that
developer can begin work immediately.

\sphinxAtStartPar
It is up to your team to decide on the definition of ready, however, a
few examples of what might be required for a task to be considered ready
for development can include (but is not limited to) the following:
\begin{enumerate}
\sphinxsetlistlabels{\arabic}{enumi}{enumii}{}{.}%
\item {} 
\sphinxAtStartPar
The story should be written exactly in the ‘user story’ format and
added to the User Stories page in the document repository. If it is
not a user story, then it does not need to follow this format nor be
added to the document repository.

\item {} 
\sphinxAtStartPar
The story should be created in the appropriate epic in the task
tracking tool and added to the backlog.

\item {} 
\sphinxAtStartPar
It must have acceptance criteria written and added to the Acceptance
Criteria page in the team’s document repository.

\item {} 
\sphinxAtStartPar
The acceptance criteria must be understood and agreed to by the team
\sphinxstyleemphasis{(we will discuss the acceptance criteria in status: in progress
below)}.

\item {} 
\sphinxAtStartPar
The team must estimate the story during a sprint planning meeting.

\item {} 
\sphinxAtStartPar
The task must have an assignee.

\item {} 
\sphinxAtStartPar
The task must have a due date.

\end{enumerate}

\sphinxAtStartPar
The following is an example of a well populated user story in Jira. It
has a title in the form of a user story, acceptance criteria, priority,
and, as it is still open, it is correctly still in the backlog.

\sphinxAtStartPar
\sphinxincludegraphics{{story_open}.png}

\sphinxAtStartPar
Status: In Progress

\sphinxAtStartPar
This status indicates that development work has commenced. For a task to
be properly in progress, it must abide by the following:
\begin{enumerate}
\sphinxsetlistlabels{\arabic}{enumi}{enumii}{}{.}%
\item {} 
\sphinxAtStartPar
It should have a single assignee. If more than one developer is
completing the work, then the task is too big, and should be broken
down into smaller, iterative tasks.

\item {} 
\sphinxAtStartPar
It should have an estimation of story points.

\item {} 
\sphinxAtStartPar
It should have a deadline when development is expected to be
completed.

\item {} 
\sphinxAtStartPar
It should be contained in a sprint.

\end{enumerate}

\sphinxAtStartPar
The following is an example of a well populated user story in Jira. It
has a title in the form of a user story, acceptance criteria, priority,
assignee, due date, and, as the story is in progress, it is assigned to
the current sprint.

\sphinxAtStartPar
\sphinxincludegraphics{{story_in_progress}.png}

\sphinxAtStartPar
Status: In Review

\sphinxAtStartPar
There are several methods that can be used to review work done
by team members.
\begin{enumerate}
\sphinxsetlistlabels{\arabic}{enumi}{enumii}{}{.}%
\item {} 
\sphinxAtStartPar
Manual review: reviewers pick up tasks that have status: in review
and review the code manually. This involves reading the code
line\sphinxhyphen{}by\sphinxhyphen{}line, running tests, etc.

\item {} 
\sphinxAtStartPar
Peer programming: Two or more developers work together on a task,
then one commits the code, and the other developers approve the code
as reviewed. This is by far the fastest way to review work and
recommended.

\item {} 
\sphinxAtStartPar
Walkthrough: As a substitute for the regular review process, a
developer can walk through their task with other teammates as part
of a collaborative process where teammates can ask questions. The
whole team is not required to be present \sphinxhyphen{} your team should decide
on a quorum required to receive approval.

\end{enumerate}

\sphinxAtStartPar
The following is an example of a well populated user story in Jira. It
has a title in the form of a user story, acceptance criteria, priority,
assignee, reviewer, due date, and, as the story is in review, it is
assigned to the current sprint.

\sphinxAtStartPar
\sphinxincludegraphics{{story_in_review}.png}

\sphinxAtStartPar
Status: Done

\sphinxAtStartPar
As well as a definition of ready, the team should have a definition of
done (DoD). The definition of done stipulates what is required before a
task can be moved to status: done.

\sphinxAtStartPar
It is up to your team to decide on the definition of done, however, a
few examples of what might be required for a task to be considered done
can include (but is not limited to) the following:
\begin{enumerate}
\sphinxsetlistlabels{\arabic}{enumi}{enumii}{}{.}%
\item {} 
\sphinxAtStartPar
Code is peer\sphinxhyphen{}reviewed.

\item {} 
\sphinxAtStartPar
Code is commented.

\item {} 
\sphinxAtStartPar
Code is checked in to trunk.

\item {} 
\sphinxAtStartPar
Code is deployed to test environment.

\item {} 
\sphinxAtStartPar
Code passes all testing listed in the team’s document repository.

\item {} 
\sphinxAtStartPar
End\sphinxhyphen{}user documentation is updated (for example, how\sphinxhyphen{}to guides).

\item {} 
\sphinxAtStartPar
Code is live on the production server.

\end{enumerate}

\sphinxAtStartPar
The following is an example of a well populated user story in Jira. It
has a title in the form of a user story, acceptance criteria, priority,
assignee, reviewer, due date, and, as the story is complete, it is
assigned to the current sprint. It also has comments from the reviewer
showing the task was reviewed with no issues found.

\sphinxAtStartPar
\sphinxincludegraphics{{story_done}.png}


\subsection{Good Hygiene}
\label{\detokenize{chapter_3/task_tracking:good-hygiene}}
\sphinxAtStartPar
Task tracking tools are common spaces shared by the entire team. They
are where teammates go to understand the status of an issue or what to
work on next, supervisors use it to monitor team progress, product
owners use it to communicate progress to the client, etc. It is
therefore imperative that they are kept in good shape:
\begin{itemize}
\item {} 
\sphinxAtStartPar
Make sure that issues are updated frequently as coding continues.

\item {} 
\sphinxAtStartPar
Add comments to describe the status of an issue as you progress it.

\item {} 
\sphinxAtStartPar
Update the deadline if it is pushed back\sphinxhyphen{} or forward.

\item {} 
\sphinxAtStartPar
Move the issue between statuses as necessary.

\end{itemize}

\begin{sphinxadmonition}{note}{Extra Resources}


\begin{savenotes}\sphinxattablestart
\centering
\begin{tabulary}{\linewidth}[t]{|T|T|}
\hline
\sphinxstyletheadfamily 
\sphinxAtStartPar
Link
&\sphinxstyletheadfamily 
\sphinxAtStartPar
Description
\\
\hline
\sphinxAtStartPar
\sphinxhref{https://www.atlassian.com/software/jira/guides/getting-started/best-practices}{Jira best practices}
&
\sphinxAtStartPar
Best practices to follow throughout the semester. It covers creating issues, sprint, etc.
\\
\hline
\sphinxAtStartPar
\sphinxhref{https://www.atlassian.com/software/jira/templates/scrum}{Jira scrum board template}
&
\sphinxAtStartPar
A template that might inspire you.
\\
\hline
\sphinxAtStartPar
\sphinxhref{https://trello.com/templates/engineering/agile-sprint-board-ZqN99gGN}{Trello board template}
&
\sphinxAtStartPar
A template that might inspire you.
\\
\hline
\end{tabulary}
\par
\sphinxattableend\end{savenotes}
\end{sphinxadmonition}


\chapter{Chapter 4: Document Repository}
\label{\detokenize{chapter_4/document_repository:chapter-4-document-repository}}\label{\detokenize{chapter_4/document_repository::doc}}
\sphinxAtStartPar
Throughout the semester, you and your team will generate a lot of
knowledge \sphinxhyphen{} in the form of software requirements, architecture, and
testing documents, etc. This knowledge will need to be centrally stored
so it is accessible to the entire team and to the client at handover.

\sphinxAtStartPar
Document repositories solve this issue by acting as a central store of
all project information and should be the first place your team
references when searching for information.

\sphinxAtStartPar
There are many options for document repository software, but in this
subject, we strongly encourage the use of Confluence.

\sphinxAtStartPar
Confluence is the industry leader and is a good tool to familiarise
yourself with, if you wish to work as a software engineer as most
companies use it.


\section{Creating Your Team’s Space}
\label{\detokenize{chapter_4/document_repository:creating-your-team-s-space}}
\sphinxAtStartPar
The first step is to create your team’s Confluence space. Only one team
member needs to do this and then grant every other team member access.

\sphinxAtStartPar
To set up the space, please follow these instructions: \sphinxhref{https://www.youtube.com/watch?v=FFF4D4I19ms\&t=259s}{Creating a space
in Confluence}.


\section{Set Up Your Space}
\label{\detokenize{chapter_4/document_repository:set-up-your-space}}
\sphinxAtStartPar
Once created, it is very important to structure your space in a way that
makes it easy to navigate.

\sphinxAtStartPar
You can begin by setting up your home page. The home page should
contain:
\begin{itemize}
\item {} 
\sphinxAtStartPar
Project summary.

\item {} 
\sphinxAtStartPar
Details of the client, supervisor, and development team.

\item {} 
\sphinxAtStartPar
Important links to other tools, for example, links to task tracking,
code repository, etc.

\end{itemize}

\sphinxAtStartPar
\sphinxincludegraphics{{home_page}.png}

\sphinxAtStartPar
Once the home page is set up, the next pages to create are meeting
pages. Over the semester, you will be required to hold team, client, and
supervisor meetings and you are expected to keep minutes for all
meetings.

\sphinxAtStartPar
So, one logical way of structuring the Confluence space could be:

\sphinxAtStartPar
\sphinxincludegraphics{{meetings}.png}

\sphinxAtStartPar
There should be a relative frequency of meetings. Supervisor meetings
should occur weekly, team meetings should occur a few times each week,
and client meetings should occur every few weeks. It is very important
you create a log of each meeting:

\sphinxAtStartPar
\sphinxincludegraphics{{meeting_structure}.png}

\sphinxAtStartPar
In the meeting page itself, your team should log:
\begin{itemize}
\item {} 
\sphinxAtStartPar
Agenda, so teammates can create questions in advance of the meeting.

\item {} 
\sphinxAtStartPar
Participants present during the meeting.

\item {} 
\sphinxAtStartPar
A video and audio recording of the meetings.

\item {} 
\sphinxAtStartPar
All discussion items and decisions reached.

\item {} 
\sphinxAtStartPar
All action items with an assigned teammate who is responsible for
the action.

\end{itemize}

\sphinxAtStartPar
\sphinxincludegraphics{{meeting_page}.png}

\begin{sphinxadmonition}{warning}{Warning:}
\sphinxAtStartPar
Throughout the semester you will be developing a
lot of resources that will need to be stored in Confluence. Please make
sure you put thought into the structure of your space. If a teammate is
unable to find work, it may result in duplication of tasks. If the
client or marker is unable to find work, they may believe you have not
done it.
\end{sphinxadmonition}


\chapter{Chapter 5: Version Control}
\label{\detokenize{chapter_5/version_control:chapter-5-version-control}}\label{\detokenize{chapter_5/version_control::doc}}
\sphinxAtStartPar
Version control is essential when collaborating on a software project.
Developers may be working on functionality within the same page or must
make changes that affect each other. Version control systems, like Git,
ensure projects can be run efficiently.

\sphinxAtStartPar
Version control is a way of tracking changes that are made to code. The
most popular version control system is Git:
\sphinxhref{https://serengetitech.com/tech/introduction-to-git-and-types-of-version-control-systems/}{more info on version control and Git}.


\section{Why Use Version Control?}
\label{\detokenize{chapter_5/version_control:why-use-version-control}}\begin{itemize}
\item {} 
\sphinxAtStartPar
Complete history of code so that any new breaking changes to
codebase can be reverted.

\item {} 
\sphinxAtStartPar
Simplifies collaboration, everyone has access to the latest version
of the codebase.

\item {} 
\sphinxAtStartPar
Improved transparency, code attributed to author.

\item {} 
\sphinxAtStartPar
Collaboration, branches can be created to work on a feature without
holding up the team.

\end{itemize}


\section{Introduction to Git}
\label{\detokenize{chapter_5/version_control:introduction-to-git}}
\sphinxAtStartPar
Git is a distributed version control system. This means that every
developer has a full copy of the repository and its history. Many
distributed version control tools use Git \sphinxhyphen{} amongst the most popular are
GitHub and BitBucket. We recommend teams use GitHub for their source
control.


\subsection{Git Terms}
\label{\detokenize{chapter_5/version_control:git-terms}}
\sphinxAtStartPar
To familiarise you with Git, below are some words you will hear repeated
often:


\begin{savenotes}\sphinxattablestart
\centering
\begin{tabulary}{\linewidth}[t]{|T|T|}
\hline
\sphinxstyletheadfamily 
\sphinxAtStartPar
Term
&\sphinxstyletheadfamily 
\sphinxAtStartPar
Definition
\\
\hline
\sphinxAtStartPar
Repository
&
\sphinxAtStartPar
Project folder which stores the project history.
\\
\hline
\sphinxAtStartPar
Remote repository
&
\sphinxAtStartPar
Version of project hosted on internet (GitHub).
\\
\hline
\sphinxAtStartPar
Local repository
&
\sphinxAtStartPar
Version of project on your machine. Changes made here are not visible by teammates unless pushed to the remote repository.
\\
\hline
\sphinxAtStartPar
Branch
&
\sphinxAtStartPar
A separate branch from the repository that can be used to make changes independent of other branches (main).
\\
\hline
\sphinxAtStartPar
Main branch
&
\sphinxAtStartPar
The default branch for your repository.
\\
\hline
\end{tabulary}
\par
\sphinxattableend\end{savenotes}


\subsection{Git Actions}
\label{\detokenize{chapter_5/version_control:git-actions}}
\sphinxAtStartPar
These are the most common actions you are likely to perform using git:


\begin{savenotes}\sphinxattablestart
\centering
\begin{tabulary}{\linewidth}[t]{|T|T|T|}
\hline
\sphinxstyletheadfamily 
\sphinxAtStartPar
Term
&\sphinxstyletheadfamily 
\sphinxAtStartPar
Definition
&\sphinxstyletheadfamily 
\sphinxAtStartPar
Command
\\
\hline
\sphinxAtStartPar
Clone
&
\sphinxAtStartPar
Makes a local copy of a repository.
&
\sphinxAtStartPar
git clone {[}repo{]}
\\
\hline
\sphinxAtStartPar
Add
&
\sphinxAtStartPar
Marks file as staged, such that is added in the next commit.
&
\sphinxAtStartPar
git add {[}filename{]}
\\
\hline
\sphinxAtStartPar
Commit
&
\sphinxAtStartPar
Snapshot of repo, with several changes to the codebase.
&
\sphinxAtStartPar
git commit \sphinxhyphen{}m “commit message”
\\
\hline
\sphinxAtStartPar
Push
&
\sphinxAtStartPar
Pushes changes (commits) to the remote repository.
&
\sphinxAtStartPar
git push
\\
\hline
\sphinxAtStartPar
Pull
&
\sphinxAtStartPar
Pulls any changes from the latest version of the remote repository and integrates any file changes with your local branch.
&
\sphinxAtStartPar
git pull {[}remote{]} {[}branch{]}
\\
\hline
\sphinxAtStartPar
Fetch
&
\sphinxAtStartPar
Pulls changes from the latest version of the remote repository but does not change the files to match remote repository. Can be useful if you want to check whether a pull will override any of your local file changes.
&
\sphinxAtStartPar
git fetch {[}remote{]} {[}branch{]}
\\
\hline
\end{tabulary}
\par
\sphinxattableend\end{savenotes}


\subsection{Local File Changes}
\label{\detokenize{chapter_5/version_control:local-file-changes}}
\sphinxAtStartPar
Changes made to files that live inside a repository can be one of
several statuses.


\begin{savenotes}\sphinxattablestart
\centering
\begin{tabulary}{\linewidth}[t]{|T|T|}
\hline
\sphinxstyletheadfamily 
\sphinxAtStartPar
Status
&\sphinxstyletheadfamily 
\sphinxAtStartPar
Description
\\
\hline
\sphinxAtStartPar
Untracked
&
\sphinxAtStartPar
The file is not being monitored by the version control (does not exist).
\\
\hline
\sphinxAtStartPar
Unmodified
&
\sphinxAtStartPar
The local file matches exactly the files in the remote repository.
\\
\hline
\sphinxAtStartPar
Modified
&
\sphinxAtStartPar
The local file differs from the file in the remote repository.
\\
\hline
\sphinxAtStartPar
Staged
&
\sphinxAtStartPar
Changes to the local file are ready to be pushed to the remote repository so the file in the remote repository matches exactly the file in the local repository.
\\
\hline
\end{tabulary}
\par
\sphinxattableend\end{savenotes}

\sphinxAtStartPar
\sphinxincludegraphics{{git_commands}.png}


\subsection{Branching}
\label{\detokenize{chapter_5/version_control:branching}}
\sphinxAtStartPar
Git branches are effectively pointers to your self\sphinxhyphen{}contained changes.
When you, as a developer, want to make changes to the code of a project
in a remote repository (no matter how big or small the change is), you
will create a branch to encapsulate all your changes. That branch is
entirely self\sphinxhyphen{}contained and is not a part of the remote repository’s
code. This stops potentially unstable code from being committed to the
repository before it can be thoroughly tested and properly merged.

\sphinxAtStartPar
To learn more about branching, please refer
\sphinxhref{https://www.atlassian.com/git/tutorials/using-branches}{here}.

\sphinxAtStartPar
To learn the basics of branching in a fun way, check out this
\sphinxhref{https://learngitbranching.js.org/}{interactive tool to learn git branching}. This tool is great for
beginners or people who need a refresher.


\subsection{Example Git Workflow}
\label{\detokenize{chapter_5/version_control:example-git-workflow}}
\sphinxAtStartPar
\sphinxstyleemphasis{Bringing everything we have discussed together}.

\sphinxAtStartPar
Once you have cloned (downloaded the most recent copy of the codebase)
from GitHub. You have what is called a \sphinxstyleemphasis{local copy} of the repository.

\sphinxAtStartPar
Then, any changes you make within your working directory (copy of the
repository on your machine) is said to be untracked. This means Git does
not know about the file and its changes.

\sphinxAtStartPar
To ensure git tracks the file and its history, you need to add the file
using the command git add {[}filename{]}.

\sphinxAtStartPar
Once a file is tracked, its changes fall under two categories: staged or
unstaged. For unstaged changes, Git has not marked the file to be a part
of the subsequent commit. Staged changes refer to files with changes
that are to be added to the next commit. When making a change that you
want to be added to the remote copy of the repository, you need to make
sure the changes are staged. This can be done by using the command git
add {[}filename{]}.

\sphinxAtStartPar
When all your changes are staged, then you want to commit those changes.
A commit is a snapshot/milestone with a series of changes. Commits can
be created with a message, using the following command: git commit \sphinxhyphen{}m
“commit message”.

\sphinxAtStartPar
Once a commit is made, the file goes back to the unmodified state as the
local repository updates the current branch’s history with the latest
changes. To ensure the remote repository is also updated, such that
everyone in the team can see the changes made, you need to push the
changes to the remote repository. This can be done by running git push.

\sphinxAtStartPar
Another developer who may want to see your changes, can pull changes by
running the command git pull {[}remote{]} {[}branch{]}. This will fetch the
latest changes from the remote repository and integrate them with their
local files.

\begin{sphinxadmonition}{note}{Extra Resources}
\begin{enumerate}
\sphinxsetlistlabels{\arabic}{enumi}{enumii}{}{.}%
\item {} 
\sphinxAtStartPar
\sphinxhref{https://www.youtube.com/watch?v=hwP7WQkmECE\&ab\_channel=Fireship}{Git explained in 100 seconds}

\item {} 
\sphinxAtStartPar
\sphinxhref{https://www.gitkraken.com}{GitKraken}: A great GUI that sits on
top of Git’s command line integration to provide a more
user\sphinxhyphen{}friendly way of interacting with version control
systems.

\end{enumerate}
\end{sphinxadmonition}

\begin{sphinxadmonition}{note}{What’s Next}

\sphinxAtStartPar
This was a very general introduction to Git. Throughout this semester,
you will be making use of GitHub \sphinxhyphen{} to learn more about GitHub, please
see the next chapter focused primarily on GitHub.
\end{sphinxadmonition}


\chapter{Chapter 6: GitHub}
\label{\detokenize{chapter_6/github:chapter-6-github}}\label{\detokenize{chapter_6/github::doc}}

\section{Configuring Your Team’s Repository}
\label{\detokenize{chapter_6/github:configuring-your-team-s-repository}}
\sphinxAtStartPar
Your first task once you decide to use GitHub is to set up a repository.
You can learn how to do that
\sphinxhref{https://docs.github.com/en/get-started/quickstart/create-a-repo}{here}.

\sphinxAtStartPar
Once you have created a repository, there are several considerations
when establishing how your team will work with the repository.


\subsection{README}
\label{\detokenize{chapter_6/github:readme}}
\sphinxAtStartPar
All teams should have a project level README file that explains your
project, and what your system aims to do. It should also provide some
detail as to how to use your code, along with any other useful
information (the directory structure could be useful for complex file
structures).

\sphinxAtStartPar
Teams can also add README files for separate components of the project,
such as frontend and backend, which may detail specific
commands/instructions.


\subsection{Licenses}
\label{\detokenize{chapter_6/github:licenses}}
\sphinxAtStartPar
If you are creating a repository with the intention of making it
publicly accessible eventually, you need to have a license to allow
other developers to collaborate (and to protect you from misuse). GitHub
provides a
\sphinxhref{https://docs.github.com/en/repositories/managing-your-repositorys-settings-and-features/customizing-your-repository/licensing-a-repository}{detailed guide on licenses}
along with a tool to determine which license is appropriate for your
repository.


\subsection{Branch Protection}
\label{\detokenize{chapter_6/github:branch-protection}}
\sphinxAtStartPar
When working on a software project, you want to ensure that critical
branches (production branches) are not accidentally tampered with or
deleted. Branch protection helps solve this by allowing developers to
configure their repository such that key branches have certain
protections that prevent deletions or require certain checks to pass
before modifications. Such branches are called \sphinxstylestrong{protected branches}.

\sphinxAtStartPar
Find out how to protect your branches on GitHub
\sphinxhref{https://docs.github.com/en/repositories/configuring-branches-and-merges-in-your-repository/defining-the-mergeability-of-pull-requests/managing-a-branch-protection-rule}{here}.


\subsection{Integrations}
\label{\detokenize{chapter_6/github:integrations}}
\sphinxAtStartPar
Integrations can help manage your development workflow. \sphinxhyphen{} Rather than
having to visit GitHub’s website each time you have a pull request, you
can make use of the
\sphinxhref{https://slack.com/intl/en-au/help/articles/232289568-GitHub-for-Slack}{GitHub bot}
on Slack to automatically send you notifications of new PRs. Discord
has a similar webhook that can be configured using this
\sphinxhref{https://gist.github.com/jagrosh/5b1761213e33fc5b54ec7f6379034a22}{guide}.


\section{Code Reviews}
\label{\detokenize{chapter_6/github:code-reviews}}
\sphinxAtStartPar
Code reviews involve your teammates reviewing your work to ensure that
code quality is maintained, and any errors are identified.


\subsection{Code Review Activities}
\label{\detokenize{chapter_6/github:code-review-activities}}\begin{itemize}
\item {} 
\sphinxAtStartPar
\sphinxstyleemphasis{Pair programming:} Having two developers working on a single unit of
work; one person writes the code (driver) whilst the other reviews
code real\sphinxhyphen{}time (navigator).

\item {} 
\sphinxAtStartPar
\sphinxstyleemphasis{Pull requests:} Having 1 or more developers review code changes
before it is merged.

\end{itemize}


\subsection{Pull Requests}
\label{\detokenize{chapter_6/github:pull-requests}}
\begin{sphinxadmonition}{note}{Note:}
\sphinxAtStartPar
Commonly referred to as PRs.
\end{sphinxadmonition}

\sphinxAtStartPar
Pull requests allow developers to review changes made on a certain
branch before merging it into another. When team members complete some
functionality and want it to be merged, they create a pull request,
requesting for their changes to be pulled into the main branch.
Developers can nominate reviewers to look at their pull request.
Reviewers can inspect the code and either approve the changes, or
request changes. Once approved, the developer can merge their code into
the main branch.


\subsubsection{Why Use Pull Requests?}
\label{\detokenize{chapter_6/github:why-use-pull-requests}}\begin{itemize}
\item {} 
\sphinxAtStartPar
Team members must review and check that the code changes are
acceptable, thereby improving code quality.

\item {} 
\sphinxAtStartPar
Errors have a greater chance of being noticed early, as there is
more than one team member reviewing/testing the changes.

\item {} 
\sphinxAtStartPar
They improve the team’s understanding of code as developers are
forced to read other developers’ code; thereby dispersing knowledge
of the code’s functionality more widely across the team.

\end{itemize}


\subsubsection{Pull Request Templates}
\label{\detokenize{chapter_6/github:pull-request-templates}}
\sphinxAtStartPar
Ensuring your pull request has enough detail for a reviewer is crucial.
If the reviewer does not understand what part of functionality has been
added, they cannot provide a meaningful review. To avoid this, a pull
request template can be useful. This provides a guideline on what
information a developer should provide in the pull request, along with
any mandatory checks that have already been completed.

\sphinxAtStartPar
You can view a sample template
\sphinxhref{https://unimelbcloud-my.sharepoint.com/personal/eduardo\_oliveira\_unimelb\_edu\_au/Documents/2022/SWEN90009/Book/assets/pull\_request\_template.md}{here}.

\sphinxAtStartPar
To see an example of how to include PR templates in your repository,
please see this
\sphinxhref{https://medium.com/@swapnesh/why-you-should-add-a-pull-request-template-in-your-github-project-1556170e55c9}{Medium article}.


\section{Git Workflows (Branching Strategies)}
\label{\detokenize{chapter_6/github:git-workflows-branching-strategies}}
\sphinxAtStartPar
A standard approach to organising Git actions within the project. Let us
consider two common strategies…


\subsection{Feature Branching}
\label{\detokenize{chapter_6/github:feature-branching}}
\sphinxAtStartPar
Feature branching entails creating a new branch for a new feature and
using that same branch until the feature is completed. After completion,
the feature branch is merged back into the main branch.

\sphinxAtStartPar
\sphinxincludegraphics{{feature_development}.png}

\sphinxAtStartPar
\sphinxstyleemphasis{Source:}
\sphinxhref{https://www.optimizely.com/optimization-glossary/trunk-based-development/}{Optimizely}

\sphinxAtStartPar
There are several flavours of feature branching but we will examine
Gitflow, by far one of the most popular branching strategies.


\subsubsection{Gitflow}
\label{\detokenize{chapter_6/github:gitflow}}
\sphinxAtStartPar
Gitflow relies on long\sphinxhyphen{}lived branches for development. There are two
main branches, main and develop, with main being the production\sphinxhyphen{}ready
codebase and develop reflecting changes to be made in the next release.
Developers create feature branches off develop, and when complete,
developers make a pull request to be merged into the develop branch.
When the team is ready for a release, then a new branch is created and
the code is tested before it is merged into the main branch.


\begin{savenotes}\sphinxattablestart
\centering
\begin{tabulary}{\linewidth}[t]{|T|T|}
\hline
\sphinxstyletheadfamily 
\sphinxAtStartPar
Advantages
&\sphinxstyletheadfamily 
\sphinxAtStartPar
Disadvantages
\\
\hline
\sphinxAtStartPar
Branches remain in a clean state.
&
\sphinxAtStartPar
Long\sphinxhyphen{}lived branches can be hard to integrate with the main branch; have diverged too much.
\\
\hline
\sphinxAtStartPar
Ideal for when multiple versions of a product are required.
&
\sphinxAtStartPar
Releases are delayed, and if there are many changes, could be highly problematic.
\\
\hline
\sphinxAtStartPar

&
\sphinxAtStartPar
Final merge from develop to the main branch can have many changes which may be overlooked by developers.
\\
\hline
\end{tabulary}
\par
\sphinxattableend\end{savenotes}


\subsection{Trunk Based Development}
\label{\detokenize{chapter_6/github:trunk-based-development}}
\sphinxAtStartPar
Trunk based development uses short\sphinxhyphen{}lived branches which are regularly
merged into the trunk, reducing any delays associated with integrating
code changes.

\sphinxAtStartPar
\sphinxincludegraphics{{trunk_development}.png}

\sphinxAtStartPar
\sphinxstyleemphasis{Source:}
\sphinxhref{https://www.optimizely.com/optimization-glossary/trunk-based-development/}{Optimizely}


\begin{savenotes}\sphinxattablestart
\centering
\begin{tabulary}{\linewidth}[t]{|T|T|}
\hline
\sphinxstyletheadfamily 
\sphinxAtStartPar
Advantages
&\sphinxstyletheadfamily 
\sphinxAtStartPar
Disadvantages
\\
\hline
\sphinxAtStartPar
Reduces likelihood of divergence from main.
&
\sphinxAtStartPar
Frequent merging can lead to breaking updates.
\\
\hline
\sphinxAtStartPar
Minimise merge conflict.
&
\sphinxAtStartPar
If slow build process, then there may be delays as people merge back into trunk.
\\
\hline
\end{tabulary}
\par
\sphinxattableend\end{savenotes}


\section{Software Releases}
\label{\detokenize{chapter_6/github:software-releases}}
\sphinxAtStartPar
When your team has completed a sprint or wants to make a new release,
there are several steps you should complete.

\sphinxAtStartPar
Releases are comprised of a tag and release notes.


\section{Tags}
\label{\detokenize{chapter_6/github:tags}}
\sphinxAtStartPar
GitHub allows developers to create tags. These provide teams to mark key
milestones in a project by creating a tag off a certain commit. Tags are
useful in releases, and you can name a tag by the associated version
number, and that way developers can easily find all previous versions of
the production\sphinxhyphen{}ready software by looking at the tags.

\sphinxAtStartPar
Tags are also necessary when creating software that is not yet
production ready when working for a client. They need to be able to
access the work completed in a sprint to see changes, but if the
application is not yet live, then a tag is used to update the client on
project status.


\section{Release Notes}
\label{\detokenize{chapter_6/github:release-notes}}
\sphinxAtStartPar
When you are creating multiple releases, it is important to explain what
key changes have been made to the product. In a commercial project,
release notes inform users of changes and clearly indicate the incentive
to upgrade to the latest version (for example, patching authentication
security bugs is a good incentive for users to migrate to the latest
version of a product). In your project, the release notes provide an
appropriate level of detail for your client to understand exactly what
features and capabilities are available in each release.


\subsection{Writing Good Release Notes}
\label{\detokenize{chapter_6/github:writing-good-release-notes}}\begin{itemize}
\item {} 
\sphinxAtStartPar
Keep it simple and avoid technical jargon.

\item {} 
\sphinxAtStartPar
Be specific about what has been implemented.

\item {} 
\sphinxAtStartPar
Group your notes logically. Creating headings for fixes,
improvements and new features may be beneficial.

\end{itemize}

\sphinxAtStartPar
For an example, please refer to this \sphinxhref{https://github.com/slack-go/slack/releases}{open\sphinxhyphen{}source project’s release notes}.


\section{Creating a Release}
\label{\detokenize{chapter_6/github:creating-a-release}}
\sphinxAtStartPar
To create a release, please refer to this
\sphinxhref{https://docs.github.com/en/repositories/releasing-projects-on-github/managing-releases-in-a-repository}{tutorial}
published by GitHub.

\begin{sphinxadmonition}{note}{Extra Resources}
\begin{itemize}
\item {} 
\sphinxAtStartPar
\sphinxhref{https://www.toptal.com/software/trunk-based-development-git-flow}{Trunk vs Gitflow development}

\item {} 
\sphinxAtStartPar
\sphinxhref{https://launchdarkly.com/blog/git-branching-strategies-vs-trunk-based-development/}{Git branching strategies}

\end{itemize}
\end{sphinxadmonition}


\chapter{Chapter 7: A Simplified Workflow}
\label{\detokenize{chapter_7/simplified_workflow:chapter-7-a-simplified-workflow}}\label{\detokenize{chapter_7/simplified_workflow::doc}}
\sphinxAtStartPar
For those unfamiliar with everything covered so far, it might not be
clear how it all ties together. The use of a code repository, task
tracker, and document repository all tie together to create a unified
and organised workflow.

\sphinxAtStartPar
Taking an example, a team that uses:
\begin{itemize}
\item {} 
\sphinxAtStartPar
Jira for task tracking.

\item {} 
\sphinxAtStartPar
GitHub for the code repository.

\item {} 
\sphinxAtStartPar
Confluence for the document repository.

\end{itemize}

\sphinxAtStartPar
Bringing all of this together, allows for a workflow that can be
monitored and scaled as the team and work grows.

\sphinxAtStartPar
\sphinxincludegraphics{{workflow2}.png}


\chapter{Chapter 8: Requirements Elicitation}
\label{\detokenize{chapter_8/requirements_elicitation:chapter-8-requirements-elicitation}}\label{\detokenize{chapter_8/requirements_elicitation::doc}}
\sphinxAtStartPar
Successful products are those that meet the needs of the client and can
be easily adopted by the target user. Building a successful product can
be difficult if teams do not understand the core business need of the
product itself.

\sphinxAtStartPar
The purpose of this section is to understand more about requirements
elicitation, the broad spectrum of tasks involved to help teams derive
their requirements. By the end of this, you should understand what
artefacts are used to represent the project specifications.


\section{What Are Requirements?}
\label{\detokenize{chapter_8/requirements_elicitation:what-are-requirements}}
\sphinxAtStartPar
Before we look at how requirements elicitation works, let us consider
what requirements are. Broadly speaking, requirements define what teams
need to implement. There are three key types of requirements that are
used in software projects:
\begin{itemize}
\item {} 
\sphinxAtStartPar
Functional requirements: These stipulate what the system should do.
The Agile Scrum framework expresses functional requirements as user
stories.

\item {} 
\sphinxAtStartPar
Non\sphinxhyphen{}functional requirements: These requirements detail any
additional constraints that specify how a system should behave. Some
examples include any safety, security, or performance requirements.

\item {} 
\sphinxAtStartPar
Emotional requirements: Requirements that detail how the target user
should feel when interacting with the system.

\end{itemize}


\section{Why Do We Need to Perform Requirements Elicitation?}
\label{\detokenize{chapter_8/requirements_elicitation:why-do-we-need-to-perform-requirements-elicitation}}
\sphinxAtStartPar
Teams are provided with a high\sphinxhyphen{}level project brief and are tasked with
extracting the project requirements. Requirements elicitation helps
teams understand what these requirements are, whilst also building a
strong understanding of the project and why it is needed.

\sphinxAtStartPar
Requirements are important as they provide teams with a clear set of
features that the team will work to complete. It is also a clear way of
demonstrating to the client, what capabilities will be delivered.

\sphinxAtStartPar
A strong understanding of the business need is equally important. As
clients may often not know exactly what they want, a strong
understanding of the project is important. When teams can understand
what the key frustrations are, they are well positioned to come up with
solutions that successfully solve the client’s pain points.

\sphinxAtStartPar
The process of requirements engineering, which is simply all the
activities required to have your detailed specifications, are discussed
below.


\subsection{0. Background Research}
\label{\detokenize{chapter_8/requirements_elicitation:background-research}}
\sphinxAtStartPar
Before you speak to your client, you should develop a general
understanding of the project to aid your team in directing the meeting
and forming helpful questions. For example, a previous year was tasked
with building a personal CRM for clients. It was important for all
students to have a good understanding of what a CRM was prior to the
initial meeting.

\sphinxAtStartPar
Additionally, teams will benefit from finding competitors within the
problem domain, to look for potential functionality that the client may
want. Understanding the client’s competitors, peers, etc. will save the
client work in having to explain all facets of their business and
industry to your team.


\subsection{1. Initial Client Meeting and Elicitation}
\label{\detokenize{chapter_8/requirements_elicitation:initial-client-meeting-and-elicitation}}

\subsubsection{Client Meeting}
\label{\detokenize{chapter_8/requirements_elicitation:client-meeting}}
\sphinxAtStartPar
The first client meeting is your chance to understand more about your
client, the project, and their expectations. During this meeting, teams
should try to establish a basic understanding of the problem.
Understanding what is wrong or missing with the current system (system
as is) is critical in making sure that you build a product that can
solve these frustrations/problems. If possible, teams should also
clarify exactly what is expected from the product: who will use this,
should the product be a mobile application or website, what other
preferences should be considered.

\sphinxAtStartPar
Make sure to schedule the meeting in advance and prepare an agenda of
questions you would like to ask ahead of time. Emailing your client with
the agenda can be beneficial to ensure they are prepared. The meeting
itself should have a facilitator and note taker assigned ahead of time,
with team members jumping in as required.


\subsubsection{Elicitation}
\label{\detokenize{chapter_8/requirements_elicitation:elicitation}}
\sphinxAtStartPar
Following your initial meeting, the team should work through the meeting
minutes to extract the key requirements. Teams should work to extract
the user goals \sphinxhyphen{} the objectives the target user of the system would want
to achieve. These correlate to user stories, an agile way of
representing requirements (see Agile methodology). Additionally, while
you are in the process of reaching this shared understanding of what
should be achieved, teams should also clearly look to define what the
scope is. If there were any features or functionality mentioned earlier
by the client, that cannot be feasibly achieved, then these should be
clearly identified.

\sphinxAtStartPar
The process of writing your requirements can be done by different
methods.


\paragraph{Motivational Modelling}
\label{\detokenize{chapter_8/requirements_elicitation:motivational-modelling}}
\sphinxAtStartPar
One popular technique is to use motivational modelling. This technique
allows the functional, non\sphinxhyphen{}functional, and emotional goals of the system
to represented in a diagram. The benefit is that diagrams are often
easier to read compared to written requirements, and emotional
requirements are also considered.

\sphinxAtStartPar
Motivational modelling is done through brainstorming a list of the key
requirements of the system and stakeholders involved. The list is called
a DO/BE/FEEL/WHO list, and should detail:
\begin{itemize}
\item {} 
\sphinxAtStartPar
DO: what your system can do? (Functional goals)

\item {} 
\sphinxAtStartPar
BE: how your system should behave? (Non\sphinxhyphen{}functional goals)

\item {} 
\sphinxAtStartPar
FEEL: how users should feel when using the system? (Emotional goals)

\item {} 
\sphinxAtStartPar
WHO: key stakeholders

\end{itemize}

\sphinxAtStartPar
This list is then converted into a hierarchical diagram, as shown below.
This is a great starting point for teams to confirm the understanding of
the project requirements, with the bottom leaves corresponding to user
stories.

\sphinxAtStartPar
\sphinxincludegraphics{{motivational_model}.jpg}

\begin{sphinxadmonition}{note}{Note:}
\sphinxAtStartPar
For more information on how to create a motivational model, see the
appendix.
\end{sphinxadmonition}


\subsection{2. Elaboration}
\label{\detokenize{chapter_8/requirements_elicitation:elaboration}}
\sphinxAtStartPar
Once the general user goals are understood, teams should work to expand
and refine the information collected. This typically involves:
\begin{itemize}
\item {} 
\sphinxAtStartPar
Verifying the motivational model is consistent with the clients
understanding and expanding as required.

\item {} 
\sphinxAtStartPar
Developing personas to match your target users.

\item {} 
\sphinxAtStartPar
Developing prototypes.

\end{itemize}


\subsubsection{Personas}
\label{\detokenize{chapter_8/requirements_elicitation:personas}}
\sphinxAtStartPar
Personas refer to fictional characters that represent the different
types of users that may use your product. They play a key role in
helping teams empathise and understand the users of the product, through
using the persona to establish whether certain design decisions are
correct.

\sphinxAtStartPar
To understand the power of a persona, consider buying a car. Different
people buy cars for different reasons. Some people are looking for a
large, reliable car for transporting their families whereas other people
look for fast and luxurious cars to fit their lifestyles. If you had to
design a car for a person, it is important to understand which category
your client falls in, to make sure the car you design is suited to their
purpose. This is how personas can be useful.

\sphinxAtStartPar
\sphinxincludegraphics{{persona}.png}

\sphinxAtStartPar
\sphinxstyleemphasis{Source:}
\sphinxhref{https://99designs.com.au/blog/business/how-to-create-user-personas/}{99Designs}

\sphinxAtStartPar
It might seem weird to use fictional characters but using real people
(like your client, a team member, etc.) could result in data privacy
breaches, as well as unrealistic use cases, if they are not well aligned
to actual user requirements.


\paragraph{Writing Personas}
\label{\detokenize{chapter_8/requirements_elicitation:writing-personas}}
\sphinxAtStartPar
A persona should have the following elements:
\begin{itemize}
\item {} 
\sphinxAtStartPar
Fictional name.

\item {} 
\sphinxAtStartPar
Job title/responsibilities.

\item {} 
\sphinxAtStartPar
Demographics (age, education).

\item {} 
\sphinxAtStartPar
Goals and tasks they are trying to achieve.

\item {} 
\sphinxAtStartPar
Any frustrations.

\end{itemize}

\sphinxAtStartPar
Below is a list of popular tools for making personas:
\begin{itemize}
\item {} 
\sphinxAtStartPar
\sphinxhref{https://marvelapp.com/}{Marvel}

\item {} 
\sphinxAtStartPar
\sphinxhref{https://xtensio.com/}{Xtensio}

\item {} 
\sphinxAtStartPar
\sphinxhref{https://www.hubspot.com/}{HubSpot}

\item {} 
\sphinxAtStartPar
\sphinxhref{https://personagenerator.com/}{PersonaGenerator}

\item {} 
\sphinxAtStartPar
\sphinxhref{https://uxpressia.com/}{UXPressia}

\end{itemize}

\sphinxAtStartPar
Personas should also be diverse in demographics and backgrounds (age,
ethnicity, education, etc.) to capture real\sphinxhyphen{}life users.


\subsubsection{Prototypes}
\label{\detokenize{chapter_8/requirements_elicitation:prototypes}}
\sphinxAtStartPar
Coding a system can be complex, and time\sphinxhyphen{}consuming. If your client
realises they do not like the design of the system, or perhaps some
interface needs to be reworked, doing this on the actual system can be
incredibly time\sphinxhyphen{}consuming. Like many projects, one of the key
constraints of your project is time. As a result, simply coding the
system after deriving your requirements increases the risk of having to
make major changes to the system midway.

\sphinxAtStartPar
Prototypes provide a solution to this problem. Teams can build a
prototype, a hand drawn or digital mock system that resembles what the
final product should look or behave like. These are much quicker to
build and allow teams to get feedback from the client much earlier,
reducing the likelihood of major changes to the system’s interface
later.


\paragraph{Classes of Prototypes}
\label{\detokenize{chapter_8/requirements_elicitation:classes-of-prototypes}}\begin{itemize}
\item {} 
\sphinxAtStartPar
\sphinxstyleemphasis{Paper prototype/Low fidelity prototype}: Hand drawing of user
interface to allow it to be rapidly designed, simulated, and tested.
These are typically black and white and don’t consider choices such as
colour palette, font, and general styling.

\item {} 
\sphinxAtStartPar
\sphinxstyleemphasis{High fidelity prototype}: A digital prototype that looks very similar
to what the final product should look like. These prototypes are made
using no\sphinxhyphen{}code programs and should determine the final design of the
system.

\end{itemize}


\subparagraph{Building a Low Fidelity Prototype}
\label{\detokenize{chapter_8/requirements_elicitation:building-a-low-fidelity-prototype}}
\sphinxAtStartPar
There are many tools that can be used for creating low fidelity
prototypes:
\begin{enumerate}
\sphinxsetlistlabels{\arabic}{enumi}{enumii}{}{.}%
\item {} 
\sphinxAtStartPar
Microsoft PowerPoint: While sometimes slow and cumbersome,
PowerPoint comes complete with several good features for drawing and
designing UI elements. \sphinxincludegraphics{{low_fidelity_prototype}.png} \sphinxstyleemphasis{This is an example of a watch UI
created through PowerPoint. User interaction elements can be
created, and a user scenario was created to take the client through
during a meeting.}

\item {} 
\sphinxAtStartPar
Miro can also be used to quickly create wireframes using their
templates: \sphinxhref{https://miro.com/templates/low-fidelity-wireframes/}{Miro wireframe templates}

\end{enumerate}


\subparagraph{Building a High\sphinxhyphen{}Fidelity Prototype}
\label{\detokenize{chapter_8/requirements_elicitation:building-a-high-fidelity-prototype}}
\sphinxAtStartPar
The benefit of using a high\sphinxhyphen{}fidelity prototype vs. a low fidelity
prototype is it allows development teams to be much more specific about
how a product will look, feel, and act without having to code it (which
can be very time\sphinxhyphen{}consuming). High fidelity prototypes also allow for
greater interactivity \sphinxhyphen{} it can be demonstrated to the client and other
users, and they can interact with the prototype to click buttons, menus,
view page transitions, etc. to gain a much better idea of how the final
product will behave. This is a great way of eliciting very specific
feedback from the client on the placement of buttons, uses of colours,
etc.

\sphinxAtStartPar
There are many tools that can be used for creating high\sphinxhyphen{}fidelity
prototypes:
\begin{enumerate}
\sphinxsetlistlabels{\arabic}{enumi}{enumii}{}{.}%
\item {} 
\sphinxAtStartPar
Figma: One of the most used tools professionally (and they offer a
free \sphinxhref{https://www.figma.com/education/}{student version})

\item {} 
\sphinxAtStartPar
Axure: Fully featured with a steep learning curve, but probably
worth it if you wish to pursue UI design professionally (and they
also offer a free \sphinxhref{https://www.axure.com/edu}{student version}).

\item {} 
\sphinxAtStartPar
Adobe XD: Far less fully featured but integrates well with other
Adobe products. They also offer student discounts for their \sphinxhref{https://www.adobe.com/au/creativecloud/buy/students.html}{full suite of products}.

\end{enumerate}


\subsection{3. Negotiation, Validation, and Specification}
\label{\detokenize{chapter_8/requirements_elicitation:negotiation-validation-and-specification}}
\sphinxAtStartPar
As the various requirements artefacts are developed, teams should be
validating these with the client. If there is some disagreement between
what the team believes is achievable and what the client wants, teams
are required to negotiate and reconcile any differences. In extreme
cases where a resolution cannot be reached, speak to your supervisor.

\sphinxAtStartPar
Running usability tests with your client is the easiest way to validate
the requirements your team has created. First, a team should create
tasks that can be completed using the prototypes \sphinxhyphen{} these tasks will
represent the full set or a subset of the user stories. For example, if
a user story is that a user logs in and views their purchase, the
high\sphinxhyphen{}fidelity prototype should permit users to simulate this user story.
It should have a success narrative and all other narratives should be
failures.

\sphinxAtStartPar
If a user can successfully log in and view their purchase history, this
is the success narrative. If they cannot, the task has been failed.

\sphinxAtStartPar
Once these tasks have been created, your team should contact the client
to organise a meeting to conduct testing. Your team should assign a test
facilitator whose job it is to ask the participant (your client) to
perform the tasks your team has created. While the participant completes
each task, the facilitator observes the participant’s behaviour and
listens for feedback.

\sphinxAtStartPar
A minute taker should be assigned to note down feedback \sphinxhyphen{} all feedback
should be welcomed from the client for things as small as the colour of
a button. The more feedback you can elicit at this stage, the more time
it will save you later when you have to make much more time\sphinxhyphen{}consuming
changes to code.

\sphinxAtStartPar
One tool that can automate this process for you is
\sphinxhref{https://help.figma.com/hc/en-us/articles/360041246514-Test-your-Figma-prototypes-with-Maze}{Maze}.
These tools typically exist with all high\sphinxhyphen{}fidelity prototypes listed
above.

\sphinxAtStartPar
Finally, once all your requirements have been finalised and the client
is happy with them, you will have your set of specifications. This
should include:
\begin{itemize}
\item {} 
\sphinxAtStartPar
User stories.

\item {} 
\sphinxAtStartPar
Motivational model.

\item {} 
\sphinxAtStartPar
Personas.

\item {} 
\sphinxAtStartPar
Prototypes.

\end{itemize}

\begin{sphinxadmonition}{note}{Extra Resources}
\begin{itemize}
\item {} 
\sphinxAtStartPar
\sphinxhref{https://maze.co/guides/usability-testing/}{A Complete Beginner’s Guide to Usability
Testing}

\end{itemize}
\end{sphinxadmonition}

\begin{sphinxadmonition}{note}{What’s Next}

\sphinxAtStartPar
Now that your team has elicited the requirements \sphinxhyphen{} it is time to start
designing the application.
\end{sphinxadmonition}


\chapter{Chapter 9: Technology Stack}
\label{\detokenize{chapter_9/technology_stack:chapter-9-technology-stack}}\label{\detokenize{chapter_9/technology_stack::doc}}
\sphinxAtStartPar
Unlike most of your other subjects, this subject does not have a
pre\sphinxhyphen{}defined technology stack you must use.

\sphinxAtStartPar
There is no right or wrong answer as to your choice of technology stacks
for your web application.

\sphinxAtStartPar
However, you should consider several factors:
\begin{itemize}
\item {} 
\sphinxAtStartPar
The desires of your client.

\item {} 
\sphinxAtStartPar
The knowledge and experience of your teammates.

\item {} 
\sphinxAtStartPar
The desire to learn something new.

\item {} 
\sphinxAtStartPar
Time and budget limitations.

\end{itemize}

\sphinxAtStartPar
Like any other decision, the decision for a technology stack requires
weighing the options and deciding as a team.

\sphinxAtStartPar
Modern web frameworks are comprised of a frontend framework, backend
framework, web server, and data persistence.

\sphinxAtStartPar
\sphinxincludegraphics{{stack}.png}

\sphinxAtStartPar
\sphinxstyleemphasis{Source:}
\sphinxhref{https://rubygarage.org/blog/technology-stack-for-web-development}{RubyGarage}

\sphinxAtStartPar
There are many choices available for tech stacks, some of the most
popular are detailed below.


\section{Front\sphinxhyphen{}End Frameworks}
\label{\detokenize{chapter_9/technology_stack:front-end-frameworks}}
\sphinxAtStartPar
Your main two choices are between HTML and CSS, and JavaScript for the
front\sphinxhyphen{}end. If you choose to use HTML and CSS,
\sphinxhref{https://getbootstrap.com/docs/5.0/getting-started/introduction/}{Bootstrap}
is a framework to take work out of development.

\sphinxAtStartPar
There are several JavaScript frameworks to help your development:
\begin{itemize}
\item {} 
\sphinxAtStartPar
\sphinxhref{https://reactjs.org/}{ReactJS}

\item {} 
\sphinxAtStartPar
\sphinxhref{https://angularjs.org/}{Angular}

\item {} 
\sphinxAtStartPar
\sphinxhref{https://vuejs.org/}{Vue.js}

\end{itemize}


\section{Back\sphinxhyphen{}End Frameworks}
\label{\detokenize{chapter_9/technology_stack:back-end-frameworks}}

\begin{savenotes}\sphinxattablestart
\centering
\begin{tabulary}{\linewidth}[t]{|T|T|}
\hline
\sphinxstyletheadfamily 
\sphinxAtStartPar
Language
&\sphinxstyletheadfamily 
\sphinxAtStartPar
Framework
\\
\hline
\sphinxAtStartPar
Python
&
\sphinxAtStartPar
\sphinxhyphen{} \sphinxhref{https://www.djangoproject.com/}{Django}\sphinxhyphen{} \sphinxhref{https://flask.palletsprojects.com/en/2.0.x/}{Flask}\sphinxhyphen{} \sphinxhref{https://pylonsproject.org/}{Pylons}
\\
\hline
\sphinxAtStartPar
Java
&
\sphinxAtStartPar
\sphinxhyphen{}\sphinxhref{https://spring.io/projects/spring-boot}{Spring Boot}
\\
\hline
\sphinxAtStartPar
Ruby
&
\sphinxAtStartPar
\sphinxhyphen{} \sphinxhref{https://rubyonrails.org/}{Ruby on Rails}
\\
\hline
\sphinxAtStartPar
PHP
&
\sphinxAtStartPar
\sphinxhyphen{} \sphinxhref{https://laravel.com/}{Laravel}
\\
\hline
\end{tabulary}
\par
\sphinxattableend\end{savenotes}


\section{Databases}
\label{\detokenize{chapter_9/technology_stack:databases}}
\sphinxAtStartPar
Your web application needs a place to store all data (user,
configuration, etc. data).


\begin{savenotes}\sphinxattablestart
\centering
\begin{tabulary}{\linewidth}[t]{|T|T|}
\hline
\sphinxstyletheadfamily 
\sphinxAtStartPar
Name
&\sphinxstyletheadfamily 
\sphinxAtStartPar
Type
\\
\hline
\sphinxAtStartPar
\sphinxhref{https://www.mysql.com/}{MySQL}
&
\sphinxAtStartPar
Relational
\\
\hline
\sphinxAtStartPar
\sphinxhref{https://www.mongodb.com/}{MongoDB}
&
\sphinxAtStartPar
Non\sphinxhyphen{}relationalDocument
\\
\hline
\sphinxAtStartPar
\sphinxhref{https://www.postgresql.org/}{PostgreSQL}
&
\sphinxAtStartPar
Relational
\\
\hline
\end{tabulary}
\par
\sphinxattableend\end{savenotes}


\section{Web Servers}
\label{\detokenize{chapter_9/technology_stack:web-servers}}
\sphinxAtStartPar
And lastly, your web application needs a web server to handle requests.
\begin{itemize}
\item {} 
\sphinxAtStartPar
\sphinxhref{https://www.nginx.com/}{Nginx}

\item {} 
\sphinxAtStartPar
\sphinxhref{https://www.apache.org/}{Apache}

\end{itemize}

\begin{sphinxadmonition}{note}{Extra Resources}
\begin{itemize}
\item {} 
\sphinxAtStartPar
\sphinxhref{https://stackshare.io/stacks}{stackshare.io} is a free website your
team can use to view the stack companies use and their reviews.

\end{itemize}
\end{sphinxadmonition}


\chapter{Chapter 10: Architecture}
\label{\detokenize{chapter_10/architecture:chapter-10-architecture}}\label{\detokenize{chapter_10/architecture::doc}}
\sphinxAtStartPar
Software architecture is the fundamental organisation of the system
under development. It includes all components, their interactions, the
environment in which they operate, and the principles used to design the
software.

\sphinxAtStartPar
Once the application’s stack is chosen, it is important to outline the
architecture of the system. Detailing the architecture serves the
purpose of assuring the client that the development team has thought
through the interactions of components and the principles that
underpinned their development. It is required information to handover to
the client at the end of semester, so they can continue development.

\begin{sphinxadmonition}{note}{Extra Resources}

\sphinxAtStartPar
In this subject, you will be learning the 4+1 architectural model. You can read the original paper
\sphinxhref{https://www.cs.ubc.ca/~gregor/teaching/papers/4+1view-architecture.pdf}{here}.
\end{sphinxadmonition}


\section{4+1 Architecture Model}
\label{\detokenize{chapter_10/architecture:architecture-model}}
\sphinxAtStartPar
There are many ways of documenting the architecture of software \sphinxhyphen{} the
one we will teach is the 4+1 architecture model.

\sphinxAtStartPar
It was originally developed in the 1990s and is used to describe the
architecture using several, concurrent views. End\sphinxhyphen{}users, developers,
system engineers, and project managers all have unique views of the
system, hence the viewpoints are used to describe it from their
perspectives.

\sphinxAtStartPar
There are five views in the 4+1 view model:

\sphinxAtStartPar
\sphinxincludegraphics{{views}.png}

\sphinxAtStartPar
\sphinxstyleemphasis{Source:}
\sphinxhref{https://medium.com/javarevisited/4-1-architectural-view-model-in-software-ec407bf27258}{Medium}


\subsection{Logical View}
\label{\detokenize{chapter_10/architecture:logical-view}}
\sphinxAtStartPar
The logical view is concerned with the functionality that the system
provides to end\sphinxhyphen{}users. Domain and class diagrams are used to represent
the logical view.

\sphinxAtStartPar
Examples of diagrams that can be used to support the logical model are
domain and database models.

\sphinxAtStartPar
\sphinxincludegraphics{{domain_model}.png}

\sphinxAtStartPar
\sphinxincludegraphics{{database_model}.png}


\subsection{Process View}
\label{\detokenize{chapter_10/architecture:process-view}}
\sphinxAtStartPar
The process view deals with the dynamic aspects of the system, explains
the system processes and how they communicate, and focuses on the run
time behaviour of the system. Sequence state diagrams are used to
represent this view.


\subsection{Development View}
\label{\detokenize{chapter_10/architecture:development-view}}
\sphinxAtStartPar
This view is represented by the package diagram and illustrates a system
from a programmer’s perspective and is concerned with software
management. To demonstrate the development view, your team could
describe the architectural goals and constraints, as well as system
diagrams, and API descriptions (if any).

\sphinxAtStartPar
\sphinxincludegraphics{{goals}.png}

\sphinxAtStartPar
\sphinxincludegraphics{{system}.png}


\subsection{Physical View}
\label{\detokenize{chapter_10/architecture:physical-view}}
\sphinxAtStartPar
The physical view depicts the system from a system engineer’s point of
view. It is concerned with the topology of software components on the
physical layer, as well as the physical connections between these
components, and it represented using the deployment diagram.

\sphinxAtStartPar
Diagrams that can support the physical view are deployment diagrams.

\sphinxAtStartPar
\sphinxincludegraphics{{deployment}.png}

\sphinxAtStartPar
\sphinxincludegraphics{{pipeline}.png}


\subsection{Scenario/Use Case View}
\label{\detokenize{chapter_10/architecture:scenario-use-case-view}}
\sphinxAtStartPar
Shows a subset of important use cases and is represented using a use
case diagram. Your team should select use case(s) of architectural
significance to demonstrate using use case description and diagram, as
well as a sequence diagram.

\sphinxAtStartPar
\sphinxincludegraphics{{use_cases_description}.png}

\sphinxAtStartPar
\sphinxincludegraphics{{use_cases_diagram}.png}


\chapter{Chapter 11: DevOps}
\label{\detokenize{chapter_11/devops:chapter-11-devops}}\label{\detokenize{chapter_11/devops::doc}}
\sphinxAtStartPar
DevOps is about integrating developer and operations teams to improve
collaboration and productivity through automating
a system’s workflows (infrastructure, deployment process etc.).

\sphinxAtStartPar
The purpose of this guide is to introduce students to DevOps and
demonstrate how it can help streamline your development workflow.


\section{Before DevOps}
\label{\detokenize{chapter_11/devops:before-devops}}
\sphinxAtStartPar
There are two teams that are core to delivering functional software:
\begin{itemize}
\item {} 
\sphinxAtStartPar
\sphinxstyleemphasis{Developer teams:} develop, build, and test features.

\item {} 
\sphinxAtStartPar
\sphinxstyleemphasis{Operations teams:} manage the change management, security,
deployment, monitoring, and feedback.

\end{itemize}

\sphinxAtStartPar
The development and operations teams previously worked in isolation \sphinxhyphen{}
the developer team worked on the product while the operations team
handled the release. This led to a lot of problems when it came to when
it came to integrate code and manage the release.

\sphinxAtStartPar
\sphinxincludegraphics{{devops}.jpg}

\sphinxAtStartPar
\sphinxstyleemphasis{Source:}
\sphinxhref{https://www.accenture.com/us-en/blogs/software-engineering-blog/shinde-development-operations-silos}{Accenture}


\section{DevOps Explained}
\label{\detokenize{chapter_11/devops:devops-explained}}
\sphinxAtStartPar
DevOps aims to break down these silos by enabling developer and
operations teams to work together. This increases collaboration and
allows teams to manage releases by using automated workflows.


\subsection{Terminology}
\label{\detokenize{chapter_11/devops:terminology}}\begin{itemize}
\item {} 
\sphinxAtStartPar
\sphinxstyleemphasis{Automation:} using technology to perform a task in a reproducible
way, such that feedback is provided on the process itself and
minimal human intervention is required.

\item {} 
\sphinxAtStartPar
\sphinxstyleemphasis{DevOps pipeline:} set of automated processes. Includes continuous
integration, continuous delivery/deployment, etc.

\end{itemize}


\section{CI/CD}
\label{\detokenize{chapter_11/devops:ci-cd}}

\subsection{What Is CI/CD?}
\label{\detokenize{chapter_11/devops:what-is-ci-cd}}
\sphinxAtStartPar
CI/CD refers to continuous integration and continuous
delivery/deployment. These are a set of practices that allow teams to
deliver software to a production environment without having to rely on
cumbersome manual processes. Through automating parts of the release
process, CI/CD allows developers to make quicker releases.


\subsection{Why CI/CD?}
\label{\detokenize{chapter_11/devops:why-ci-cd}}\begin{enumerate}
\sphinxsetlistlabels{\arabic}{enumi}{enumii}{}{.}%
\item {} 
\sphinxAtStartPar
Fast: the time taken for releases are significantly reduced through
automating the workflow.

\item {} 
\sphinxAtStartPar
Simple: the process of integrating code becomes much simpler

\item {} 
\sphinxAtStartPar
Fewer errors: less intervention from developers is required which
reduces the chance for human error

\item {} 
\sphinxAtStartPar
Isolated failures: immediate feedback provided as to which step in
the pipeline has failed

\end{enumerate}

\sphinxAtStartPar
There are several CI/CD providers, but we recommend using GitHub Actions
as it is free and easy to use. When you have created your repository,
refer to our CI/CD guide on how to get started with GitHub Actions.


\subsection{CI/CD Terminology}
\label{\detokenize{chapter_11/devops:ci-cd-terminology}}\begin{itemize}
\item {} 
\sphinxAtStartPar
Continuous Integration: Integrating your code changes back to your
main branch. Changes are checked by automatically building, testing
code.

\item {} 
\sphinxAtStartPar
Continuous Delivery: Extension of CI. Automates the release process
so that you can deploy the application at any time.

\item {} 
\sphinxAtStartPar
Continuous Deployment: Extension of continuous delivery. Automate
deployment to production environment if all other stages are
successful.

\end{itemize}

\sphinxAtStartPar
\sphinxincludegraphics{{cicd}.png}

\sphinxAtStartPar
\sphinxstyleemphasis{Source:}
\sphinxhref{https://www.atlassian.com/continuous-delivery/principles/continuous-integration-vs-delivery-vs-deployment}{Atlassian}


\section{DevOps Pipeline}
\label{\detokenize{chapter_11/devops:devops-pipeline}}
\sphinxAtStartPar
Teams should use CI/CD within their projects, using either continuous
delivery or deployment. CI/CD is achieved using pipelines.


\subsection{Creating Your Pipeline}
\label{\detokenize{chapter_11/devops:creating-your-pipeline}}
\sphinxAtStartPar
Pipelines refer to a set of steps executed when some condition is met.
Often, teams will customise their pipelines to be triggered on pull
requests. This means, when a developer creates a pull request to merge
code changes from their branch to the main branch, the pipeline will
automatically run the set of steps specified.

\sphinxAtStartPar
Ordinarily, developers perform several checks before merging any code.
You want to make sure your project builds successfully, the code follows
any formatting conventions, and that all the test cases written still
pass. When creating your pipeline, these kinds of checks are written as
steps that you want to be executed on your given condition. As such,
teams should ensure that when they create their pipeline, they should
include the following steps:
\begin{itemize}
\item {} 
\sphinxAtStartPar
install dependencies and build

\item {} 
\sphinxAtStartPar
format

\item {} 
\sphinxAtStartPar
run

\item {} 
\sphinxAtStartPar
test

\end{itemize}


\subsection{An Example CI/CD Workflow}
\label{\detokenize{chapter_11/devops:an-example-ci-cd-workflow}}
\sphinxAtStartPar
Let us consider an example of how this may look. Once you have your
pipeline, the workflow is simple. Consider the case where your pipeline
is triggered on pull requests. A team member makes a pull request from
their own branch, dev/authentication, into the main branch, main. This
will trigger the pipeline to execute, and it will sequentially build,
format, run, and test as stipulated in the pipeline. If any step fails,
it will indicate which step failed and developers can even inspect the
logs to find why.

\sphinxAtStartPar
Once all the steps successfully complete, the pipeline will indicate
that the code has passed all its checks. This will then allow other
developers to review the code themselves, before approving the pull
request. Once the code is accepted, a developer can merge the code from
dev/authentication into main. From here, you can either have a separate
pipeline set up to handle the deployment process or allow your
deployment provider to deploy from the main branch.

\sphinxAtStartPar
This workflow is shown below:

\sphinxAtStartPar
\sphinxincludegraphics{{example_workflow}.png}

\sphinxAtStartPar
\sphinxstyleemphasis{Source:}
\sphinxhref{https://docs.gitlab.com/ee/ci/introduction/}{Gitlab}


\section{Deployment}
\label{\detokenize{chapter_11/devops:deployment}}
\sphinxAtStartPar
Deployment refers to the set of steps that are required to \sphinxstyleemphasis{make your
application accessible to the public}. There are three types of
deployment tools available:


\begin{savenotes}\sphinxattablestart
\centering
\begin{tabulary}{\linewidth}[t]{|T|T|T|}
\hline
\sphinxstyletheadfamily 
\sphinxAtStartPar
SaaS
&\sphinxstyletheadfamily 
\sphinxAtStartPar
PaaS
&\sphinxstyletheadfamily 
\sphinxAtStartPar
IaaS
\\
\hline
\sphinxAtStartPar
Software as a service.
&
\sphinxAtStartPar
Platform as a service.
&
\sphinxAtStartPar
Infrastructure as a service.
\\
\hline
\sphinxAtStartPar
The vendor manages everything.
&
\sphinxAtStartPar
Vendor manages architecture, but you develop the application and manage data.
&
\sphinxAtStartPar
You manage the infrastructure, data, and application.
\\
\hline
\sphinxAtStartPar

&
\sphinxAtStartPar

&
\sphinxAtStartPar

\\
\hline
\end{tabulary}
\par
\sphinxattableend\end{savenotes}

\begin{sphinxadmonition}{note}{Note:}
\sphinxAtStartPar
For this subject, we highly recommend you use PaaS deployment tools
as this allows you to focus on building the application.
\end{sphinxadmonition}


\section{PaaS}
\label{\detokenize{chapter_11/devops:paas}}

\subsection{PaaS Providers}
\label{\detokenize{chapter_11/devops:paas-providers}}\begin{itemize}
\item {} 
\sphinxAtStartPar
\sphinxhref{https://www.heroku.com/}{Heroku}

\item {} 
\sphinxAtStartPar
\sphinxhref{https://aws.amazon.com/elasticbeanstalk/}{AWS elastic beanstalk}

\item {} 
\sphinxAtStartPar
\sphinxhref{https://www.netlify.com/}{Netlify}

\item {} 
\sphinxAtStartPar
\sphinxhref{https://azure.microsoft.com/en-us/services/cloud-services/}{Microsoft Azure Cloud Services}

\item {} 
\sphinxAtStartPar
\sphinxhref{https://pages.github.com/}{GitHub Pages}

\end{itemize}


\subsection{PaaS Deployment}
\label{\detokenize{chapter_11/devops:paas-deployment}}
\sphinxAtStartPar
Deployment could be achieved in several ways and is up to your team to
choose a deployment strategy:
\begin{itemize}
\item {} 
\sphinxAtStartPar
Configure PaaS deployment tool to deploy automatically from your
main branch.

\item {} 
\sphinxAtStartPar
Specify the deployment as a step in your pipeline.

\end{itemize}

\begin{sphinxadmonition}{note}{Extra Resources}
\begin{itemize}
\item {} 
\sphinxAtStartPar
\sphinxhref{https://www.youtube.com/watch?v=scEDHsr3APg\&ab\_channel=Fireship}{Short summary of CI/CD}

\item {} 
\sphinxAtStartPar
\sphinxhref{https://www.bmc.com/blogs/saas-vs-paas-vs-iaas-whats-the-difference-and-how-to-choose/}{PaaS vs SaaS vs IaaS}

\end{itemize}
\end{sphinxadmonition}


\chapter{Chapter 12: Adding CI/CD to Your Project}
\label{\detokenize{chapter_12/ci_cd:chapter-12-adding-ci-cd-to-your-project}}\label{\detokenize{chapter_12/ci_cd::doc}}
\begin{sphinxadmonition}{note}{Note:}
\sphinxAtStartPar
There are many tools available to create a CI/CD pipeline. This
guide will demonstrate using GitHub Actions.
\end{sphinxadmonition}


\section{GitHub Actions}
\label{\detokenize{chapter_12/ci_cd:github-actions}}

\begin{savenotes}\sphinxattablestart
\centering
\begin{tabulary}{\linewidth}[t]{|T|T|}
\hline
\sphinxstyletheadfamily 
\sphinxAtStartPar
Key Terms
&\sphinxstyletheadfamily 
\sphinxAtStartPar
Meaning
\\
\hline
\sphinxAtStartPar
Workflows
&
\sphinxAtStartPar
Automated procedure that can be triggered by some event. Written in YAML.
\\
\hline
\sphinxAtStartPar
Events
&
\sphinxAtStartPar
An activity that triggers a workflow to be executed. For example, pushes to main branch.
\\
\hline
\sphinxAtStartPar
Jobs
&
\sphinxAtStartPar
Set of steps that execute on the same runner. Multiple jobs run in parallel.
\\
\hline
\sphinxAtStartPar
Steps
&
\sphinxAtStartPar
Individual task that runs command within a job. Executes shell commands or actions. Multiple steps run sequentially.
\\
\hline
\sphinxAtStartPar
Actions
&
\sphinxAtStartPar
Standalone commands that are combined into steps. There are actions published by the Github community which you can use as well.
\\
\hline
\sphinxAtStartPar
Runner
&
\sphinxAtStartPar
The server that the workflows run within.
\\
\hline
\end{tabulary}
\par
\sphinxattableend\end{savenotes}


\subsection{Create Your First Workflow}
\label{\detokenize{chapter_12/ci_cd:create-your-first-workflow}}\begin{enumerate}
\sphinxsetlistlabels{\arabic}{enumi}{enumii}{}{.}%
\item {} 
\sphinxAtStartPar
Create a directory from the root of your repository called
.github/workflows.

\item {} 
\sphinxAtStartPar
Create a new file for your workflow. You can name this anything, but
it must be a YAML file. For example, ci\sphinxhyphen{}cd.yaml.

\item {} 
\sphinxAtStartPar
Add the following example workflow and modify/add steps as required.

\end{enumerate}

\begin{sphinxVerbatim}[commandchars=\\\{\}]
\PYG{n}{name}\PYG{o}{*}\PYG{o}{*}\PYG{p}{:}\PYG{o}{*}\PYG{o}{*} \PYG{n}{learn}\PYG{o}{\PYGZhy{}}\PYG{n}{github}\PYG{o}{\PYGZhy{}}\PYG{n}{actions}\PYGZbs{}
\PYG{n}{on}\PYG{o}{*}\PYG{o}{*}\PYG{p}{:}\PYG{o}{*}\PYG{o}{*} \PYG{o}{*}\PYG{o}{*}\PYGZbs{}\PYG{p}{[}\PYG{o}{*}\PYG{o}{*}\PYG{n}{push}\PYG{o}{*}\PYG{o}{*}\PYGZbs{}\PYG{p}{]}\PYG{o}{*}\PYG{o}{*}\PYGZbs{}
\PYG{n}{jobs}\PYG{o}{*}\PYG{o}{*}\PYG{p}{:}\PYG{o}{*}\PYG{o}{*}\PYGZbs{}
\PYG{n}{check}\PYG{o}{\PYGZhy{}}\PYG{n}{bats}\PYG{o}{\PYGZhy{}}\PYG{n}{version}\PYG{o}{*}\PYG{o}{*}\PYG{p}{:}\PYG{o}{*}\PYG{o}{*}\PYGZbs{}
\PYG{n}{runs}\PYG{o}{\PYGZhy{}}\PYG{n}{on}\PYG{o}{*}\PYG{o}{*}\PYG{p}{:}\PYG{o}{*}\PYG{o}{*} \PYG{n}{ubuntu}\PYG{o}{\PYGZhy{}}\PYG{n}{latest}\PYGZbs{}
\PYG{n}{steps}\PYG{o}{*}\PYG{o}{*}\PYG{p}{:}\PYG{o}{*}\PYG{o}{*}\PYGZbs{}
\PYG{o}{*}\PYG{o}{*}\PYG{o}{\PYGZhy{}}\PYG{o}{*}\PYG{o}{*} \PYG{n}{uses}\PYG{o}{*}\PYG{o}{*}\PYG{p}{:}\PYG{o}{*}\PYG{o}{*} \PYG{n}{actions}\PYG{o}{/}\PYG{n}{checkout}\PYG{n+nd}{@v2}\PYGZbs{}
\PYG{o}{*}\PYG{o}{*}\PYG{o}{\PYGZhy{}}\PYG{o}{*}\PYG{o}{*} \PYG{n}{uses}\PYG{o}{*}\PYG{o}{*}\PYG{p}{:}\PYG{o}{*}\PYG{o}{*} \PYG{n}{actions}\PYG{o}{/}\PYG{n}{setup}\PYG{o}{\PYGZhy{}}\PYG{n}{node}\PYG{n+nd}{@v2}\PYGZbs{}
\PYG{k}{with}\PYG{o}{*}\PYG{o}{*}\PYG{p}{:}\PYG{o}{*}\PYG{o}{*}\PYGZbs{}
\PYG{n}{node}\PYG{o}{\PYGZhy{}}\PYG{n}{version}\PYG{o}{*}\PYG{o}{*}\PYG{p}{:}\PYG{o}{*}\PYG{o}{*} \PYGZbs{}\PYG{l+s+s1}{\PYGZsq{}}\PYG{l+s+s1}{14}\PYG{l+s+se}{\PYGZbs{}\PYGZsq{}}\PYG{l+s+se}{\PYGZbs{}}
\PYG{l+s+s1}{**\PYGZhy{}** run**:** npm install \PYGZhy{}g bats}\PYG{l+s+se}{\PYGZbs{}}
\PYG{l+s+s1}{**\PYGZhy{}** run**:** bats \PYGZhy{}v}
\end{sphinxVerbatim}

\sphinxAtStartPar
The steps above were obtained from the GitHub Actions
\sphinxhref{https://docs.github.com/en/actions/learn-github-actions/understanding-github-actions\#create-an-example-workflow}{documentation}.


\subsection{Understanding the Workflow Syntax}
\label{\detokenize{chapter_12/ci_cd:understanding-the-workflow-syntax}}
\sphinxAtStartPar
\sphinxincludegraphics{{workflow}.png}


\subsection{Viewing Your Workflow Execution}
\label{\detokenize{chapter_12/ci_cd:viewing-your-workflow-execution}}
\sphinxAtStartPar
Once your workflow is triggered, you should be able to see its execution
under the Actions tab in your GitHub repository. If any step fails,
GitHub will clearly indicate the step that failed and show any logs that
can be useful to understand why it has failed.


\subsection{Steps to Include in Your Pipeline}
\label{\detokenize{chapter_12/ci_cd:steps-to-include-in-your-pipeline}}\begin{enumerate}
\sphinxsetlistlabels{\arabic}{enumi}{enumii}{}{.}%
\item {} 
\sphinxAtStartPar
Build your application
\begin{enumerate}
\sphinxsetlistlabels{\arabic}{enumii}{enumiii}{}{.}%
\item {} 
\sphinxAtStartPar
Install necessary dependencies

\end{enumerate}

\item {} 
\sphinxAtStartPar
Perform any formatting/linting checks

\item {} 
\sphinxAtStartPar
Test your application
\begin{enumerate}
\sphinxsetlistlabels{\arabic}{enumii}{enumiii}{}{.}%
\item {} 
\sphinxAtStartPar
Unit testing

\item {} 
\sphinxAtStartPar
Integration testing

\end{enumerate}

\end{enumerate}

\begin{sphinxadmonition}{note}{Extra Resources}

\sphinxAtStartPar
To build your first project using GitHub Actions, you can take
\sphinxhref{https://lab.github.com/githubtraining/github-actions:-hello-world}{GitHub’s free course}.

\sphinxAtStartPar
You can watch this explainer in a free \sphinxhref{https://www.youtube.com/watch?v=5MJRtldPOEI}{YouTube video}.
\end{sphinxadmonition}


\chapter{Chapter 13: Testing}
\label{\detokenize{chapter_13/testing:chapter-13-testing}}\label{\detokenize{chapter_13/testing::doc}}
\sphinxAtStartPar
Testing is an important part of developing software.

\sphinxAtStartPar
In your project, you will be required to test your project, along with
documenting your testing objectives, processes, and strategies within
Confluence.

\sphinxAtStartPar
This will introduce the different types of testing to students and
explain how to write a test plan.


\section{Why Write Tests?}
\label{\detokenize{chapter_13/testing:why-write-tests}}
\sphinxAtStartPar
Software testing is the process of verifying that the software you have
developed meets the requirements of the client.

\sphinxAtStartPar
The benefits of testing include:
\begin{itemize}
\item {} 
\sphinxAtStartPar
Identify bugs in your software.

\item {} 
\sphinxAtStartPar
Identify regression failures (checking if new code breaks existing
features).

\item {} 
\sphinxAtStartPar
Reduce reliance on developers to test different cases.

\item {} 
\sphinxAtStartPar
Automate tests is much faster than running test scenarios by hand.

\end{itemize}


\section{Types of Testing}
\label{\detokenize{chapter_13/testing:types-of-testing}}
\sphinxAtStartPar
Each type of testing will be discussed in further detail below.


\begin{savenotes}\sphinxattablestart
\centering
\begin{tabulary}{\linewidth}[t]{|T|T|}
\hline
\sphinxstyletheadfamily 
\sphinxAtStartPar
Type
&\sphinxstyletheadfamily 
\sphinxAtStartPar
Description
\\
\hline
\sphinxAtStartPar
Unit testing
&
\sphinxAtStartPar
Checking whether each software unit performs as expected. The finest level of granularity in testing.
\\
\hline
\sphinxAtStartPar
Integration testing
&
\sphinxAtStartPar
Checking whether multiple software components/units operate correctly.
\\
\hline
\sphinxAtStartPar
Acceptance testing
&
\sphinxAtStartPar
Checking whether the application meets the client’s requirements. Acceptance tests are created based on the user stories.
\\
\hline
\end{tabulary}
\par
\sphinxattableend\end{savenotes}


\subsection{Unit Testing}
\label{\detokenize{chapter_13/testing:unit-testing}}
\sphinxAtStartPar
Unit testing is completed during the development of an application and
is completed by the developers. When a developer picks up a task from
the task tracking tool, they are expected to implement the task as
required and prove that the unit of work does what they say it does.
This is where unit testing comes in.

\sphinxAtStartPar
Unit tests isolate a section of code to verify correctness. A unit may
be an individual function, method, procedure, module, or object.


\subsubsection{Reasons to Complete Unit Testing}
\label{\detokenize{chapter_13/testing:reasons-to-complete-unit-testing}}\begin{enumerate}
\sphinxsetlistlabels{\arabic}{enumi}{enumii}{}{.}%
\item {} 
\sphinxAtStartPar
Unit tests help to fix bugs early in the development cycle when it
is less expensive and easier to locate bugs.

\item {} 
\sphinxAtStartPar
Good unit tests serve as project documentation.

\item {} 
\sphinxAtStartPar
Unit tests permit code re\sphinxhyphen{}use \sphinxhyphen{} if you know a component you have
created is thoroughly tested and works well, you can introduce it
into other parts of your application’s codebase without being
concerned it will cause issues.

\end{enumerate}

\sphinxAtStartPar
There is a tension that exists in software testing between available
time to implement a task and time that should be spent on testing each
task. Many developers cut down on testing time to complete more work.
This is a myth that you can somehow minimise time spent testing and end
up with a fully functional and bug\sphinxhyphen{}free piece of software.

\sphinxAtStartPar
It in fact saves time unit testing tasks as they are completed. They are
easy to isolate \sphinxhyphen{} introducing a buggy component into a larger piece of
work is akin to finding a needle in a haystack when bugs arise later
(and bugs will arise later, if you have not tested them).


\subsubsection{Types of Unit Testing}
\label{\detokenize{chapter_13/testing:types-of-unit-testing}}
\sphinxAtStartPar
There are two primary types of unit testing: manual and automated.


\paragraph{Manual Unit Testing}
\label{\detokenize{chapter_13/testing:manual-unit-testing}}
\sphinxAtStartPar
As the name suggests, this involves manually testing a unit of work.
This is normally done by the developer who completed the work and
involves creating documentation detailing the step\sphinxhyphen{}by\sphinxhyphen{}step instructions
(with screenshots and comments) showing the success scenarios of the
unit, as well as any failure scenarios.

\sphinxAtStartPar
This documentation should be tested and confirmed by a reviewer (another
member of the development team).


\paragraph{Automated Unit Testing}
\label{\detokenize{chapter_13/testing:automated-unit-testing}}
\sphinxAtStartPar
There are several tools that can be used to automate unit testing
(depending on the language chosen for your project). Some examples are:
\begin{enumerate}
\sphinxsetlistlabels{\arabic}{enumi}{enumii}{}{.}%
\item {} 
\sphinxAtStartPar
\sphinxhref{https://junit.org/junit5/}{JUnit}: a popular and free
testing tool for Java.

\item {} 
\sphinxAtStartPar
\sphinxhref{https://jestjs.io}{Jest}: a popular and free testing
tool for JavaScript.

\item {} 
\sphinxAtStartPar
\sphinxhref{https://testing-library.com/}{Testing library}.

\item {} 
\sphinxAtStartPar
\sphinxhref{https://mochajs.org}{Mocha}: a popular and free
testing tool for JavaScript.

\end{enumerate}


\subsection{Integration Testing}
\label{\detokenize{chapter_13/testing:integration-testing}}
\sphinxAtStartPar
Integration testing involves integrating software modules and testing
them as a group.

\sphinxAtStartPar
You might wonder why integration testing is necessary when all tasks
have (hopefully) been unit tested:
\begin{enumerate}
\sphinxsetlistlabels{\arabic}{enumi}{enumii}{}{.}%
\item {} 
\sphinxAtStartPar
A unit is designed by one developer whose understanding and
programming logic may differ from other developers. Integration
testing verifies that software modules work in unity regardless of
which developer developed the module.

\item {} 
\sphinxAtStartPar
Changes of requirements by the clients could mean some modules are
outdated and need to be tested and integrated carefully.

\item {} 
\sphinxAtStartPar
Calls and return statements between software modules and the
database, hosting service, etc. could be erroneous.

\item {} 
\sphinxAtStartPar
Exception handling between components could be lacking.

\end{enumerate}

\sphinxAtStartPar
Integration testing differs from other tests as it primarily focuses on
information and data flow between modules. The emphasis is on
integrating links between modules.

\sphinxAtStartPar
As an example, you might have two modules:
\begin{enumerate}
\sphinxsetlistlabels{\arabic}{enumi}{enumii}{}{.}%
\item {} 
\sphinxAtStartPar
Authentication component

\item {} 
\sphinxAtStartPar
Email component.

\end{enumerate}

\sphinxAtStartPar
Both modules work as expected when conducting unit testing, but now the
team must test that information flows from authentication to the email
module. An example test case might be:


\begin{savenotes}\sphinxattablestart
\centering
\begin{tabulary}{\linewidth}[t]{|T|T|T|T|}
\hline
\sphinxstyletheadfamily 
\sphinxAtStartPar
Test Case ID
&\sphinxstyletheadfamily 
\sphinxAtStartPar
Objective
&\sphinxstyletheadfamily 
\sphinxAtStartPar
Description
&\sphinxstyletheadfamily 
\sphinxAtStartPar
Expected Result
\\
\hline
\sphinxAtStartPar
1
&
\sphinxAtStartPar
Test the link between the authentication and email modules.
&
\sphinxAtStartPar
Enter login and credentials and click login button.
&
\sphinxAtStartPar
User should be successfully redirected to the email module.
\\
\hline
\end{tabulary}
\par
\sphinxattableend\end{savenotes}

\sphinxAtStartPar
This is just one example of testing the integration between two
components \sphinxhyphen{} when conducting actual integration testing, there would
likely be many more tests.


\subsubsection{Automated Integration Testing}
\label{\detokenize{chapter_13/testing:automated-integration-testing}}
\sphinxAtStartPar
There are several tools that can be used to automate integration testing
(depending on the language chosen for your project). Some examples are:
\begin{enumerate}
\sphinxsetlistlabels{\arabic}{enumi}{enumii}{}{.}%
\item {} 
\sphinxAtStartPar
\sphinxhref{https://playwright.dev/}{Playwright}

\item {} 
\sphinxAtStartPar
\sphinxhref{https://www.cypress.io/}{Cypress}

\end{enumerate}


\subsection{Acceptance Testing}
\label{\detokenize{chapter_13/testing:acceptance-testing}}
\sphinxAtStartPar
Acceptance testing tests whether the system meets the requirements as
specified by the client \sphinxhyphen{} it tests whether the system passes or fails a
given user story.

\sphinxAtStartPar
Often, the process of acceptance testing begins during the requirements
phase. Teams write acceptance criteria to describe a pass/fail set of
criteria that is used to determine whether a user story is completed.
Acceptance criteria do not care about how you implement the
functionality.

\sphinxAtStartPar
Once you have your acceptance criteria, you can use this to write test
cases that are used to verify the result of the criteria (acceptance
tests).


\subsubsection{Acceptance Criteria Template}
\label{\detokenize{chapter_13/testing:acceptance-criteria-template}}
\sphinxAtStartPar
When writing acceptance criteria, you should use the template below:

\sphinxAtStartPar
Given that {[}condition{]}, when {[}something happens{]}, then {[}result{]}.


\subsubsection{Acceptance Criteria Vs. Acceptance Tests}
\label{\detokenize{chapter_13/testing:acceptance-criteria-vs-acceptance-tests}}

\begin{savenotes}\sphinxattablestart
\centering
\begin{tabulary}{\linewidth}[t]{|T|T|}
\hline
\sphinxstyletheadfamily 
\sphinxAtStartPar
Acceptance Criteria
&\sphinxstyletheadfamily 
\sphinxAtStartPar
Acceptance Test
\\
\hline
\sphinxAtStartPar
Measures if a user story is completed.
&
\sphinxAtStartPar
Verifies if the product works for users.
\\
\hline
\sphinxAtStartPar
Written before implementation.
&
\sphinxAtStartPar
Written during development.
\\
\hline
\sphinxAtStartPar
Describes what areas to cover in the project.
&
\sphinxAtStartPar
Uses criteria to test key areas of the project.
\\
\hline
\end{tabulary}
\par
\sphinxattableend\end{savenotes}

\sphinxAtStartPar
As an example, here is a list of the acceptance criteria for a project
from semester 2, 2021:

\sphinxAtStartPar
\sphinxincludegraphics{{acceptance_criteria}.png}

\sphinxAtStartPar
You can see in the screenshot above that the acceptance criteria links
to: epic, user story, and details the criteria for a pass result.

\sphinxAtStartPar
This criterion was then test using acceptance tests and the results were
manually captured in screenshots and added to the Confluence page:

\sphinxAtStartPar
\sphinxincludegraphics{{acceptance_testing}.png}

\sphinxAtStartPar
The details in the acceptance test (screenshots + steps involved) allow
the tests to be easily reproduced by the development team, client, or
academic staff in the future.

\begin{sphinxadmonition}{note}{Note:}\begin{itemize}
\item {} 
\sphinxAtStartPar
Your team should have 1\sphinxhyphen{}3 acceptance criteria per user story.

\item {} 
\sphinxAtStartPar
Acceptance criteria are written before implementation.

\item {} 
\sphinxAtStartPar
They are independently testable.

\item {} 
\sphinxAtStartPar
They have a defined pass or fail result.

\end{itemize}
\end{sphinxadmonition}


\subsection{Visualising Testing}
\label{\detokenize{chapter_13/testing:visualising-testing}}
\sphinxAtStartPar
The hierarchy of testing can be visualised as:

\sphinxAtStartPar
\sphinxincludegraphics{{testing_hierarchy}.svg}


\section{Test Plan}
\label{\detokenize{chapter_13/testing:test-plan}}
\sphinxAtStartPar
Test plans document the team’s testing strategy, objectives, tools, and
processes for the project.

\sphinxAtStartPar
Testing plans are important as they ensure:
\begin{itemize}
\item {} 
\sphinxAtStartPar
All team members understand the objectives the team must strive to
achieve.

\item {} 
\sphinxAtStartPar
Ensures testing is not neglected throughout the project.

\item {} 
\sphinxAtStartPar
Forces the team to think about how they will verify the quality of
their product.

\end{itemize}


\subsection{What to Include in a Test Plan?}
\label{\detokenize{chapter_13/testing:what-to-include-in-a-test-plan}}\begin{itemize}
\item {} 
\sphinxAtStartPar
Testing objectives.

\item {} 
\sphinxAtStartPar
Testing tools/frameworks.

\item {} 
\sphinxAtStartPar
In scope/Out of scope (what aspects of the system will/won’t be
tested).

\item {} 
\sphinxAtStartPar
A clear description of what types of testing will be completed.
Should provide detail as to who will be conducting the testing and
when testing will occur.

\item {} 
\sphinxAtStartPar
Bug reporting strategy (when a bug is identified, what process
should the team follow?).

\end{itemize}

\sphinxAtStartPar
You can view a very thorough example of a test plan \sphinxhref{https://www.softwaretestinghelp.com/test-plan-sample-softwaretesting-and-quality-assurance-templates/}{here}.


\section{Code Coverage}
\label{\detokenize{chapter_13/testing:code-coverage}}
\sphinxAtStartPar
A common metric for thoroughness of your test plan is code coverage. It
examines what proportion of code is covered by a test. While code
coverage is not an indicator of success (i.e., coverage scores of 100\%
does not necessarily mean you have tested the right things), it is a
good indicator of test extensiveness. Code coverage score greater than
80\% is generally considered sufficient.

\sphinxAtStartPar
Code coverage works by testing many or all the below:
\begin{enumerate}
\sphinxsetlistlabels{\arabic}{enumi}{enumii}{}{.}%
\item {} 
\sphinxAtStartPar
\sphinxstyleemphasis{Function coverage:} how many of the functions defined have been
called.

\item {} 
\sphinxAtStartPar
\sphinxstyleemphasis{Statement coverage:} how many of the statements in the program have
been executed.

\item {} 
\sphinxAtStartPar
\sphinxstyleemphasis{Branches coverage:} how many of the branches of the control
structures (if statements for instance) have been executed.

\item {} 
\sphinxAtStartPar
\sphinxstyleemphasis{Condition coverage:} how many of the Boolean sub\sphinxhyphen{}expressions have
been tested for a true and a false value.

\item {} 
\sphinxAtStartPar
\sphinxstyleemphasis{Line coverage:} how many of lines of source code have been tested.

\end{enumerate}

\sphinxAtStartPar
Code coverage reports are created in the process by tools (detailed
below) and are displayed:

\sphinxAtStartPar
\sphinxincludegraphics{{code_coverage}.png}

\sphinxAtStartPar
\sphinxstyleemphasis{Source}: \sphinxhref{https://www.atlassian.com/continuous-delivery/software-testing/code-coverage}{Atlassian}


\subsection{Automating Code Coverage}
\label{\detokenize{chapter_13/testing:automating-code-coverage}}
\sphinxAtStartPar
There are many tools that can automate code coverage, but the two most
popular are:
\begin{enumerate}
\sphinxsetlistlabels{\arabic}{enumi}{enumii}{}{.}%
\item {} 
\sphinxAtStartPar
\sphinxhref{https://docs.cypress.io/guides/tooling/code-coverage\#Introduction}{Cypress}.

\item {} 
\sphinxAtStartPar
\sphinxhref{https://github.com/gotwarlost/istanbul}{Istanbul}.

\end{enumerate}


\chapter{Chapter 14: Final Presentation}
\label{\detokenize{chapter_14/final_presentation:chapter-14-final-presentation}}\label{\detokenize{chapter_14/final_presentation::doc}}
\sphinxAtStartPar
The final presentation is your opportunity to showcase your product to
supervisors, other teams, and your client. Each team will have
approximately 15 minutes to walk through their project and demo their
application.


\section{What Should Be Covered?}
\label{\detokenize{chapter_14/final_presentation:what-should-be-covered}}
\sphinxAtStartPar
Your team’s presentation should cover the following (in no order):
\begin{itemize}
\item {} 
\sphinxAtStartPar
Introduction to your team: who you are, what you did.

\item {} 
\sphinxAtStartPar
Business need: what drove the initial requirement for the product
you developed.

\item {} 
\sphinxAtStartPar
Project requirements: how did you work with your client to develop
requirements.

\item {} 
\sphinxAtStartPar
Target users/personas: who will use the application you have
developed.

\item {} 
\sphinxAtStartPar
Project process:
\begin{itemize}
\item {} 
\sphinxAtStartPar
The design of the system.

\item {} 
\sphinxAtStartPar
An overview of your application’s architecture.

\end{itemize}

\item {} 
\sphinxAtStartPar
Project outcomes: what was delivered;

\item {} 
\sphinxAtStartPar
Application demonstration:
\begin{itemize}
\item {} 
\sphinxAtStartPar
This should be a demonstration of your fully deployed
application.

\item {} 
\sphinxAtStartPar
Develop a narrative for your demonstration and utilise personas
to complete tasks using the application.

\end{itemize}

\item {} 
\sphinxAtStartPar
Key challenges and accomplishments.

\end{itemize}

\begin{sphinxadmonition}{attention}{Attention:}
\sphinxAtStartPar
The presentations should be easy to follow by the audience, and
you should assume your audience has no prior knowledge of your
application (you will likely be marked by academic staff that have never
seen your work prior to the demonstration).
\end{sphinxadmonition}


\section{Presenters}
\label{\detokenize{chapter_14/final_presentation:presenters}}
\sphinxAtStartPar
\sphinxstylestrong{All} team members are required to speak during the presentation.


\section{Tools to Take Your Presentation to the Next Level}
\label{\detokenize{chapter_14/final_presentation:tools-to-take-your-presentation-to-the-next-level}}
\sphinxAtStartPar
Whilst PowerPoint is a great for making slide decks, the templates
provided are limited. To make your presentation more professional, we
encourage you to find templates online that match the branding of your
application. We recommend you look at the following websites for modern
PowerPoint templates:
\begin{itemize}
\item {} 
\sphinxAtStartPar
\sphinxhref{https://www.canva.com/en\_au/}{Canva}

\item {} 
\sphinxAtStartPar
\sphinxhref{https://www.adobe.com/express/}{Adobe Creative Cloud Express}

\end{itemize}


\section{During the Demonstration}
\label{\detokenize{chapter_14/final_presentation:during-the-demonstration}}\begin{itemize}
\item {} 
\sphinxAtStartPar
\sphinxstyleemphasis{Develop tasks and use personas to complete the tasks:} Teams are
required to demonstrate their application during their presentation.
We recommend walking through your application as if you were the
target user. Employing personas can help make the presentation more
engaging.

\item {} 
\sphinxAtStartPar
\sphinxstyleemphasis{Demonstrate high value features}: As you decide what features to
showcase, remember that you should show the functionality that is of
high value to your client. You need to show how the application you
built meets your client’s expectations. Activities such as
authentication can be omitted as it rarely is a specific requirement
from the client. You can use the user stories priorities to decide
what is of high value to the client.

\item {} 
\sphinxAtStartPar
\sphinxstyleemphasis{Use realistic data:} When demonstrating, it is important to use
realistic data where possible. You are trying to show how a
real\sphinxhyphen{}life customer would use the application, and so the data should
be realistic, too. We recommend teams to have their mock data
written or saved somewhere so the demonstrator knows exactly what to
use.

\item {} 
\sphinxAtStartPar
\sphinxstyleemphasis{Preparing for failure is also important as things can go wrong
during a demonstration.} Sometimes the application may not be
predictable, or you may discover a bug mid\sphinxhyphen{}demonstration. Practicing
can help reduce the likelihood of this occurring, but teams should
still have a plan for what to do in case of failure. Having an
authenticated session ready as a backup can be useful, or having
another student stand\sphinxhyphen{}by to take\sphinxhyphen{}over, if one student is having
issues.

\end{itemize}

\begin{sphinxadmonition}{tip}{Tip:}\begin{itemize}
\item {} 
\sphinxAtStartPar
Hold practice presentation to iron out timing and transitions.

\item {} 
\sphinxAtStartPar
Practice presenting to a friend or family member who has no
background knowledge of the project \sphinxhyphen{} if they can follow along and
understand your demonstration, that is a good sign.

\item {} 
\sphinxAtStartPar
Have a practice run through on the day.

\item {} 
\sphinxAtStartPar
Practice the timing of your presentation.

\item {} 
\sphinxAtStartPar
Keep your slides succinct, as overloading slides with text can make
them more difficult to follow \sphinxhyphen{} try replacing text with graphics
where it makes sense.

\item {} 
\sphinxAtStartPar
If the presentations are being held virtually:
\begin{itemize}
\item {} 
\sphinxAtStartPar
Use a consistent virtual background as a team.

\item {} 
\sphinxAtStartPar
Add your group name to your Zoom handle.

\item {} 
\sphinxAtStartPar
Make sure your audio and video are clear.

\end{itemize}

\item {} 
\sphinxAtStartPar
Practice how you will transition between speakers.

\end{itemize}
\end{sphinxadmonition}


\chapter{Chapter 15: Client Handover}
\label{\detokenize{chapter_15/client_handover:chapter-15-client-handover}}\label{\detokenize{chapter_15/client_handover::doc}}
\sphinxAtStartPar
At the end of the semester, you are required to handover your web application to your client in a state that will
allow them to continue development. That means they must be given accurate information and develop an understanding of
the current state of the application.

\sphinxAtStartPar
You are required to hold a meeting with the client in which you will discuss and demonstrate the final delivered
application and the user stories implemented (and those not implemented).
In the meeting you should handover a tag of the code repository, which should include:
\begin{itemize}
\item {} 
\sphinxAtStartPar
All code developed throughout the semester.

\item {} 
\sphinxAtStartPar
The correct code license with ownership transferred to the client.

\item {} 
\sphinxAtStartPar
Populated READMEs with a clear description of the repository structure.

\item {} 
\sphinxAtStartPar
Details on how to access the database and the credentials required to access it.

\item {} 
\sphinxAtStartPar
Details on how to access the hosting service and the credentials required to access it.

\item {} 
\sphinxAtStartPar
Details of all data stored in the database (this includes user data used for testing, etc.).

\item {} 
\sphinxAtStartPar
An export of the entire document repository including all work completed throughout the semester.

\item {} 
\sphinxAtStartPar
A manual showing the use cases of the system (you can use your acceptance testing here).

\item {} 
\sphinxAtStartPar
Instruction on how to start\sphinxhyphen{}up and shutdown the application.

\end{itemize}


\part{Appendices}


\chapter{Appendix A: Extra Tools}
\label{\detokenize{appendices/appendix_a/extra_tools:appendix-a-extra-tools}}\label{\detokenize{appendices/appendix_a/extra_tools::doc}}
\sphinxAtStartPar
This is a collection of some of the academic staff’s favourite tools.


\section{Version Control Tools}
\label{\detokenize{appendices/appendix_a/extra_tools:version-control-tools}}

\subsection{Git Graphical User Interfaces (GUIs)}
\label{\detokenize{appendices/appendix_a/extra_tools:git-graphical-user-interfaces-guis}}
\begin{sphinxadmonition}{note}{Note:}
\sphinxAtStartPar
A graphical user interface is an interface that permits interaction with git system commands in an easier way.
They are especially good for visualising code changes in large repositories or for developers new to git.
\end{sphinxadmonition}


\begin{savenotes}\sphinxattablestart
\centering
\begin{tabulary}{\linewidth}[t]{|T|}
\hline
\sphinxstyletheadfamily 
\sphinxAtStartPar
Tool
\\
\hline
\sphinxAtStartPar
\sphinxhref{https://www.gitkraken.com}{GitKraken}
\\
\hline
\sphinxAtStartPar
\sphinxhref{https://www.sourcetreeapp.com}{Sourcetree}
\\
\hline
\end{tabulary}
\par
\sphinxattableend\end{savenotes}


\section{Integrated Development Environment (IDE) Tools}
\label{\detokenize{appendices/appendix_a/extra_tools:integrated-development-environment-ide-tools}}
\begin{sphinxadmonition}{note}{Note:}
\sphinxAtStartPar
An integrated development environment is a software application that provides comprehensive facilities to computer
programmers for software development. An IDE normally consists of at least a source code editor, build automation
tools and a debugger.
\end{sphinxadmonition}


\subsection{IDEs}
\label{\detokenize{appendices/appendix_a/extra_tools:ides}}

\begin{savenotes}\sphinxattablestart
\centering
\begin{tabulary}{\linewidth}[t]{|T|T|}
\hline
\sphinxstyletheadfamily 
\sphinxAtStartPar
IDE
&\sphinxstyletheadfamily 
\sphinxAtStartPar
Development Type
\\
\hline
\sphinxAtStartPar
\sphinxhref{https://code.visualstudio.com}{VSCode}
&
\sphinxAtStartPar
Web
\\
\hline
\sphinxAtStartPar
\sphinxhref{https://www.jetbrains.com/community/education/\#students}{IntelliJ}
&
\sphinxAtStartPar
Java
\\
\hline
\sphinxAtStartPar
\sphinxhref{https://www.jetbrains.com/community/education/\#students}{PyCharm}
&
\sphinxAtStartPar
Python
\\
\hline
\sphinxAtStartPar
\sphinxhref{https://www.jetbrains.com/community/education/\#students}{Webstorm}
&
\sphinxAtStartPar
Web
\\
\hline
\sphinxAtStartPar
\sphinxhref{https://www.eclipse.org}{Eclipse}
&
\sphinxAtStartPar
Java
\\
\hline
\sphinxAtStartPar
\sphinxhref{https://developer.android.com/studio}{Android Studio}
&
\sphinxAtStartPar
Android
\\
\hline
\sphinxAtStartPar
\sphinxhref{https://developer.apple.com/xcode/}{XCode}
&
\sphinxAtStartPar
iOS
\\
\hline
\end{tabulary}
\par
\sphinxattableend\end{savenotes}


\subsubsection{VSCode Plugins}
\label{\detokenize{appendices/appendix_a/extra_tools:vscode-plugins}}

\begin{savenotes}\sphinxattablestart
\centering
\begin{tabulary}{\linewidth}[t]{|T|T|}
\hline
\sphinxstyletheadfamily 
\sphinxAtStartPar
Plugin
&\sphinxstyletheadfamily 
\sphinxAtStartPar
Description
\\
\hline
\sphinxAtStartPar
\sphinxhref{https://marketplace.visualstudio.com/items?itemName=esbenp.prettier-vscode}{Prettier}
&
\sphinxAtStartPar
Code formatting.
\\
\hline
\sphinxAtStartPar
\sphinxhref{https://marketplace.visualstudio.com/items?itemName=eamodio.gitlens}{Gitlens}
&
\sphinxAtStartPar
Git visualisation.
\\
\hline
\sphinxAtStartPar
\sphinxhref{https://marketplace.visualstudio.com/items?itemName=CoenraadS.bracket-pair-colorizer}{Bracket Pair Colorizer}
&
\sphinxAtStartPar
Colours brackets to more easily distinguish.
\\
\hline
\sphinxAtStartPar
\sphinxhref{https://marketplace.visualstudio.com/items?itemName=MS-vsliveshare.vsliveshare}{Live share}
&
\sphinxAtStartPar
Great for pair programming.
\\
\hline
\sphinxAtStartPar
\sphinxhref{https://marketplace.visualstudio.com/items?itemName=dbaeumer.vscode-eslint}{ESLint}
&
\sphinxAtStartPar
Linting.
\\
\hline
\sphinxAtStartPar
\sphinxhref{https://marketplace.visualstudio.com/items?itemName=streetsidesoftware.code-spell-checker}{Spell checker}
&
\sphinxAtStartPar
Spell checker that works with camel case.
\\
\hline
\end{tabulary}
\par
\sphinxattableend\end{savenotes}


\subsubsection{IntelliJ Plugins}
\label{\detokenize{appendices/appendix_a/extra_tools:intellij-plugins}}

\begin{savenotes}\sphinxattablestart
\centering
\begin{tabulary}{\linewidth}[t]{|T|T|}
\hline
\sphinxstyletheadfamily 
\sphinxAtStartPar
Plugin
&\sphinxstyletheadfamily 
\sphinxAtStartPar
Description
\\
\hline
\sphinxAtStartPar
\sphinxhref{https://www.jetbrains.com/help/idea/code-with-me.html}{Code With Me}
&
\sphinxAtStartPar

\\
\hline
\sphinxAtStartPar
\sphinxhref{https://plugins.jetbrains.com/plugin/10080-rainbow-brackets}{Rainbow Brackets}
&
\sphinxAtStartPar
Colorises brackets to easily distinguish.
\\
\hline
\sphinxAtStartPar
\sphinxhref{https://plugins.jetbrains.com/plugin/7017-plantuml-integration}{PlantUML integration}
&
\sphinxAtStartPar
A plugin to create diagrams.
\\
\hline
\sphinxAtStartPar
\sphinxhref{https://plugins.jetbrains.com/plugin/11938-one-dark-theme}{One Dark Theme}
&
\sphinxAtStartPar

\\
\hline
\end{tabulary}
\par
\sphinxattableend\end{savenotes}


\section{Communication Tools}
\label{\detokenize{appendices/appendix_a/extra_tools:communication-tools}}

\subsection{Slack Plugins}
\label{\detokenize{appendices/appendix_a/extra_tools:slack-plugins}}

\begin{savenotes}\sphinxattablestart
\centering
\begin{tabulary}{\linewidth}[t]{|T|T|}
\hline
\sphinxstyletheadfamily 
\sphinxAtStartPar
Tool
&\sphinxstyletheadfamily 
\sphinxAtStartPar
Description
\\
\hline
\sphinxAtStartPar
\sphinxhref{https://geekbot.com/}{Geekbot}
&
\sphinxAtStartPar
Virtual stand\sphinxhyphen{}ups within slack.
\\
\hline
\end{tabulary}
\par
\sphinxattableend\end{savenotes}


\chapter{Appendix B: Motivational Model Guide}
\label{\detokenize{appendices/appendix_b/motivational_model_guide:appendix-b-motivational-model-guide}}\label{\detokenize{appendices/appendix_b/motivational_model_guide::doc}}
\begin{sphinxadmonition}{attention}{Attention:}
\sphinxAtStartPar
An announcement will be posted in the Canvas LMS when the MMTool is
available to access for your subject. No further registration is
required.
\end{sphinxadmonition}

\sphinxAtStartPar
Login to the MMTool using:
\begin{itemize}
\item {} 
\sphinxAtStartPar
URL: provided during the semester.

\item {} 
\sphinxAtStartPar
Username: your University of Melbourne email address.

\item {} 
\sphinxAtStartPar
Password: your University of Melbourne password.

\end{itemize}

\begin{sphinxadmonition}{important}{Important:}
\sphinxAtStartPar
If you face any issues while logging using the provided information,
please email the teaching team and include the below details:
\begin{itemize}
\item {} 
\sphinxAtStartPar
Student ID

\item {} 
\sphinxAtStartPar
First Name

\item {} 
\sphinxAtStartPar
Last Name

\item {} 
\sphinxAtStartPar
Email Address

\item {} 
\sphinxAtStartPar
Username

\item {} 
\sphinxAtStartPar
Description of the issue and if required include corresponding
screenshots

\end{itemize}
\end{sphinxadmonition}


\section{Creating a Model}
\label{\detokenize{appendices/appendix_b/motivational_model_guide:creating-a-model}}
\sphinxAtStartPar
Once the project teams are finalised, teaching team will create a
project including all the team members of your team in the MMTool. An
announcement will be posted in the Canvas LMS.

\sphinxAtStartPar
All the members of a team should be able to view the project in your
dashboard as shown in Figure 1.

\sphinxAtStartPar
\sphinxincludegraphics{{image1}.png}

\begin{sphinxadmonition}{note}{Note:}
\sphinxAtStartPar
Project Name (Team Name) should not be renamed.
\end{sphinxadmonition}


\subsection{Creating a Model Under a Project}
\label{\detokenize{appendices/appendix_b/motivational_model_guide:creating-a-model-under-a-project}}\begin{enumerate}
\sphinxsetlistlabels{\arabic}{enumi}{enumii}{}{.}%
\item {} 
\sphinxAtStartPar
Click on ‘+’ under ‘Create Model’ in Figure 1 to create a new model.

\item {} 
\sphinxAtStartPar
Enter model name in the displayed ‘New Model’ window and click
‘Create’:

\end{enumerate}

\sphinxAtStartPar
\sphinxincludegraphics{{image2}.png}
\begin{enumerate}
\sphinxsetlistlabels{\arabic}{enumi}{enumii}{}{.}%
\item {} 
\sphinxAtStartPar
On clicking Create in the New Model window, user will be navigated
to the Write Goal List tab of the Model.

\end{enumerate}

\sphinxAtStartPar
\sphinxincludegraphics{{image3}.png}
\begin{enumerate}
\sphinxsetlistlabels{\arabic}{enumi}{enumii}{}{.}%
\item {} 
\sphinxAtStartPar
To Rename the model:

\end{enumerate}
\begin{itemize}
\item {} 
\sphinxAtStartPar
Click on stack button corresponding to a model in project folder:

\end{itemize}

\sphinxAtStartPar
\sphinxincludegraphics{{image4}.png}
\begin{itemize}
\item {} 
\sphinxAtStartPar
Enter the model’s name in ‘Model Rename’ window and Click ‘Confirm’:

\end{itemize}

\sphinxAtStartPar
\sphinxincludegraphics{{image5}.png}


\subsection{Steps to Create a Motivational Model}
\label{\detokenize{appendices/appendix_b/motivational_model_guide:steps-to-create-a-motivational-model}}
\sphinxAtStartPar
MMTool consists of three tabs:
\begin{itemize}
\item {} 
\sphinxAtStartPar
Write Goal List: to enter Do, Be, Feel, Concern and Who goals.

\item {} 
\sphinxAtStartPar
Cluster Goals/Arrange into Hierarchy: to create cluster in the
right pane by dragging and dropping the goals from
Do/Be/Feel/Concern/Who lists from left pane.

\item {} 
\sphinxAtStartPar
Render Model: to render model from the cluster created.

\end{itemize}


\subsection{Step 1: Write Goal List}
\label{\detokenize{appendices/appendix_b/motivational_model_guide:step-1-write-goal-list}}
\sphinxAtStartPar
The motivational model uses four notations to denote the goals
associated with the project:
\begin{itemize}
\item {} 
\sphinxAtStartPar
Parallelogram to represent a functional goal.

\item {} 
\sphinxAtStartPar
Cloud to represent a quality goal.

\item {} 
\sphinxAtStartPar
Heart to represent an emotional goal (Inverted heart to represent
concern).

\item {} 
\sphinxAtStartPar
Person figure to represent the users/stakeholders.

\end{itemize}

\sphinxAtStartPar
Steps to create Do/Be/Feel/Who lists in the MMTool:
\begin{enumerate}
\sphinxsetlistlabels{\arabic}{enumi}{enumii}{}{.}%
\item {} 
\sphinxAtStartPar
Do/Be/Feel/Who lists developed on paper during requirements
elicitation can be uploaded in the left pane of Write Goal List tab.

\item {} 
\sphinxAtStartPar
Click ‘Upload Image’ button and select the image to upload from your
personal device:

\end{enumerate}

\sphinxAtStartPar
\sphinxincludegraphics{{image6}.png}
\begin{enumerate}
\sphinxsetlistlabels{\arabic}{enumi}{enumii}{}{.}%
\item {} 
\sphinxAtStartPar
Enter New Functional Goal under Do in the right pane of Write Goal
List tab.

\end{enumerate}

\sphinxAtStartPar
\sphinxincludegraphics{{image7}.png}
\begin{enumerate}
\sphinxsetlistlabels{\arabic}{enumi}{enumii}{}{.}%
\item {} 
\sphinxAtStartPar
User will be able to enter required number of goals by pressing
Enter on keyboard.

\end{enumerate}

\sphinxAtStartPar
\sphinxincludegraphics{{image8}.png}
\begin{enumerate}
\sphinxsetlistlabels{\arabic}{enumi}{enumii}{}{.}%
\item {} 
\sphinxAtStartPar
Click on ‘Be’ in the right pane to enter New Quality Goal:

\end{enumerate}

\sphinxAtStartPar
\sphinxincludegraphics{{image9}.png}
\begin{enumerate}
\sphinxsetlistlabels{\arabic}{enumi}{enumii}{}{.}%
\item {} 
\sphinxAtStartPar
Similarly, click on ‘Feel’ to enter the emotional goals, ‘Concern’
to enter negative goals and ‘Who’ to enter the roles associated with
the project.

\item {} 
\sphinxAtStartPar
\sphinxstyleemphasis{To Delete the uploaded image}: Click \sphinxstylestrong{‘X’} in the top\sphinxhyphen{}right
corner of the uploaded image:

\end{enumerate}

\sphinxAtStartPar
\sphinxincludegraphics{{image10}.png}
\begin{enumerate}
\sphinxsetlistlabels{\arabic}{enumi}{enumii}{}{.}%
\item {} 
\sphinxAtStartPar
\sphinxstyleemphasis{To Delete the entered goals}: Place the cursor on the goal that
needs to be deleted and click delete icon corresponding to the goal
in the list:

\end{enumerate}

\sphinxAtStartPar
\sphinxincludegraphics{{image11}.png}
\begin{enumerate}
\sphinxsetlistlabels{\arabic}{enumi}{enumii}{}{.}%
\item {} 
\sphinxAtStartPar
\sphinxstyleemphasis{To Rename the entered goals}: Click on the corresponding goal in
the list to get the cursor and rename the goal:

\end{enumerate}

\sphinxAtStartPar
\sphinxincludegraphics{{image12}.png}


\subsection{Step 2: Cluster Goals/Arrange into Hierarchy}
\label{\detokenize{appendices/appendix_b/motivational_model_guide:step-2-cluster-goals-arrange-into-hierarchy}}\begin{enumerate}
\sphinxsetlistlabels{\arabic}{enumi}{enumii}{}{.}%
\item {} 
\sphinxAtStartPar
Click on ‘Cluster Goals/Arrange into Hierarchy’ tab.

\item {} 
\sphinxAtStartPar
The lists created in the ‘Write Goal List’ tab are displayed on the
left pane in the ‘Cluster Goals/Arrange into Hierarchy’ tab:

\end{enumerate}

\sphinxAtStartPar
\sphinxincludegraphics{{image13}.png}
\begin{enumerate}
\sphinxsetlistlabels{\arabic}{enumi}{enumii}{}{.}%
\item {} 
\sphinxAtStartPar
Select a goal to drag and drop into ‘Drag here to create a new
cluster’ area of the tab:

\end{enumerate}

\sphinxAtStartPar
\sphinxincludegraphics{{image14}.png}
\begin{enumerate}
\sphinxsetlistlabels{\arabic}{enumi}{enumii}{}{.}%
\item {} 
\sphinxAtStartPar
When a goal is dropped in the ‘Drag here to create a new cluster’
area, it is by default placed at the bottom of the hierarchy. Click
on the goal in the hierarchy and re\sphinxhyphen{}drag it to the required level.

\item {} 
\sphinxAtStartPar
Goals aligned in the same column will be on the same level in the
rendered model in step 3.

\end{enumerate}

\sphinxAtStartPar
To create a hierarchy,
\begin{itemize}
\item {} 
\sphinxAtStartPar
Considering a model as a tree, the major goal of project is
represented as a root node and is placed at the top of the
hierarchy.

\item {} 
\sphinxAtStartPar
The high\sphinxhyphen{}level functional requirements (functional goals) of the
project can be further divided into subtrees and are aligned right
at one level under the root node.

\end{itemize}

\begin{sphinxadmonition}{tip}{Tip:}
\sphinxAtStartPar
Click on the goal and move the goal slightly to the right side. Once
the goal is moved to the right position, a hyphen, \sphinxhyphen{}, will appear
denoting the level of subtree.
\end{sphinxadmonition}

\sphinxAtStartPar
\sphinxincludegraphics{{image15}.png}
\begin{itemize}
\item {} 
\sphinxAtStartPar
Activities related to each subtree are further extended to detail
the associated goals and are right aligned at one level to the
subtree.

\item {} 
\sphinxAtStartPar
Click on the goal and move the goal slightly to the right side. Once
the goal is moved to the right position, a hyphen, \sphinxhyphen{}, will appear
denoting the level of subtree.

\item {} 
\sphinxAtStartPar
The goals can be re\sphinxhyphen{}dragged to the upper level of the hierarchy by
clicking on the goal and moving them to the left.

\item {} 
\sphinxAtStartPar
All the goals in the lists can be dragged and dropped at the
required level in the **’**Drag here to create a new cluster’ area.

\end{itemize}

\sphinxAtStartPar
\sphinxincludegraphics{{image16}.png}
\begin{enumerate}
\sphinxsetlistlabels{\arabic}{enumi}{enumii}{}{.}%
\item {} 
\sphinxAtStartPar
The goals in do/be/feel/concern/who lists can be edited in ‘Write
Goal List’, ‘Cluster Goals/Arrange into Hierarchy’ and ‘Render
Model’ tabs of MMTool.

\end{enumerate}


\subsection{Step 3: Render Model}
\label{\detokenize{appendices/appendix_b/motivational_model_guide:step-3-render-model}}\begin{enumerate}
\sphinxsetlistlabels{\arabic}{enumi}{enumii}{}{.}%
\item {} 
\sphinxAtStartPar
Click on the ‘Render Model’ tab to render the model:

\end{enumerate}

\sphinxAtStartPar
\sphinxincludegraphics{{image17}.png}
\begin{enumerate}
\sphinxsetlistlabels{\arabic}{enumi}{enumii}{}{.}%
\item {} 
\sphinxAtStartPar
Click on ‘Render’ button in the top\sphinxhyphen{}right corner of ‘Render Model’
tab to render the motivational model from the hierarchy created in
the Step 2.

\item {} 
\sphinxAtStartPar
The model rendered is:

\end{enumerate}

\sphinxAtStartPar
\sphinxincludegraphics{{image18}.png}
\begin{enumerate}
\sphinxsetlistlabels{\arabic}{enumi}{enumii}{}{.}%
\item {} 
\sphinxAtStartPar
The shapes may overlap in the rendered model. Model needs to be
adjusted manually to make it presentable and clear:

\end{enumerate}

\sphinxAtStartPar
\sphinxincludegraphics{{image19}.png}
\begin{enumerate}
\sphinxsetlistlabels{\arabic}{enumi}{enumii}{}{.}%
\item {} 
\sphinxAtStartPar
Click on ‘Save’ button present at the top to save the model:

\end{enumerate}

\sphinxAtStartPar
\sphinxincludegraphics{{image20}.png}
\begin{enumerate}
\sphinxsetlistlabels{\arabic}{enumi}{enumii}{}{.}%
\item {} 
\sphinxAtStartPar
Click on ‘Export’ button at the top\sphinxhyphen{}right corner to export the
rendered model:

\end{enumerate}

\sphinxAtStartPar
\sphinxincludegraphics{{image21}.png}
\begin{enumerate}
\sphinxsetlistlabels{\arabic}{enumi}{enumii}{}{.}%
\item {} 
\sphinxAtStartPar
In the rendered model:

\end{enumerate}
\begin{itemize}
\item {} 
\sphinxAtStartPar
The notations (parallelogram, cloud, heart, and person) used in the
model can be expandable and shrinkable.

\item {} 
\sphinxAtStartPar
The text in the notations is editable.

\item {} 
\sphinxAtStartPar
The position of the notations is adjustable.

\end{itemize}


\section{Writing Notes}
\label{\detokenize{appendices/appendix_b/motivational_model_guide:writing-notes}}
\sphinxAtStartPar
Click on ‘Notes’ button in the top\sphinxhyphen{}right corner of the model to write
any information related to the team discussion as required:

\sphinxAtStartPar
\sphinxincludegraphics{{image22}.png}


\section{Returning to the Dashboard}
\label{\detokenize{appendices/appendix_b/motivational_model_guide:returning-to-the-dashboard}}
\sphinxAtStartPar
Click on the ‘Return’ button in the top\sphinxhyphen{}right corner of the model to
return to the dashboard at any point of time:

\sphinxAtStartPar
\sphinxincludegraphics{{image23}.png}


\section{View Profile}
\label{\detokenize{appendices/appendix_b/motivational_model_guide:view-profile}}\begin{enumerate}
\sphinxsetlistlabels{\arabic}{enumi}{enumii}{}{.}%
\item {} 
\sphinxAtStartPar
Click on username present in the top\sphinxhyphen{}right corner:

\end{enumerate}

\sphinxAtStartPar
\sphinxincludegraphics{{image24}.png}
\begin{enumerate}
\sphinxsetlistlabels{\arabic}{enumi}{enumii}{}{.}%
\item {} 
\sphinxAtStartPar
Click on \sphinxstylestrong{‘Profile’} button to view your profile details:

\end{enumerate}

\sphinxAtStartPar
\sphinxincludegraphics{{image25}.png}


\section{Signing Out}
\label{\detokenize{appendices/appendix_b/motivational_model_guide:signing-out}}\begin{enumerate}
\sphinxsetlistlabels{\arabic}{enumi}{enumii}{}{.}%
\item {} 
\sphinxAtStartPar
Click on username present in the top\sphinxhyphen{}right corner:

\end{enumerate}

\sphinxAtStartPar
\sphinxincludegraphics{{image24}.png}
\begin{enumerate}
\sphinxsetlistlabels{\arabic}{enumi}{enumii}{}{.}%
\item {} 
\sphinxAtStartPar
Click on ‘Sign out’ button to sign out from the tool:

\end{enumerate}

\sphinxAtStartPar
\sphinxincludegraphics{{image25}.png}


\chapter{Appendix C: Tools for Personas}
\label{\detokenize{appendices/appendix_c/personas_guide:appendix-c-tools-for-personas}}\label{\detokenize{appendices/appendix_c/personas_guide::doc}}
\sphinxAtStartPar
There are different tools you can use to create your personas. In this tutorial, we will introduce 4 tools to you.


\section{Hubspot \sphinxhyphen{} Make My Persona}
\label{\detokenize{appendices/appendix_c/personas_guide:hubspot-make-my-persona}}\begin{itemize}
\item {} 
\sphinxAtStartPar
Firstly, you do NOT need a Hubspot account to use this tool. Visit
\sphinxhref{https://www.hubspot.com/make-my-persona}{this link}.

\item {} 
\sphinxAtStartPar
Once you click the “Build My Persona” button, it will take you through a 7\sphinxhyphen{}steps tour to create your persona.

\item {} 
\sphinxAtStartPar
Or, you can just choose to skip walkthrough mode and edit it yourself.
\sphinxincludegraphics{{MakeMyPersona1}.jpg} \sphinxincludegraphics{{MakeMyPersona2}.JPG}

\item {} 
\sphinxAtStartPar
Either way, you will get the chance to edit/add/delete sections at the end.

\end{itemize}


\subsection{Save and Export}
\label{\detokenize{appendices/appendix_c/personas_guide:save-and-export}}
\sphinxAtStartPar
Yes \sphinxhyphen{} you can save this persona and edit it later. What you do is:
\begin{enumerate}
\sphinxsetlistlabels{\arabic}{enumi}{enumii}{}{.}%
\item {} 
\sphinxAtStartPar
Once you finish the 7\sphinxhyphen{}steps tour, or you have chosen the option to skip the tour, you will arrive at a
“Make My Persona Overview” page.

\item {} 
\sphinxAtStartPar
Click the save button.
\sphinxincludegraphics{{MakeMyPersona3}.JPG}

\item {} 
\sphinxAtStartPar
It will ask you to fill in some personal details.

\item {} 
\sphinxAtStartPar
After you have filled in your personal details, click the “Download Now” button.
\sphinxincludegraphics{{MakeMyPersona4}.JPG}

\item {} 
\sphinxAtStartPar
You can download your persona as a PDF. You will also be given a link that you can use to access it in the
future, or share with your team members
\sphinxincludegraphics{{MakeMyPersona5}.JPG}

\end{enumerate}


\subsection{Any Disadvantages?}
\label{\detokenize{appendices/appendix_c/personas_guide:any-disadvantages}}
\sphinxAtStartPar
For the persona photo, you can only use a pre\sphinxhyphen{}defined set of avatar (around 15 choices) \sphinxhyphen{} no real images options.
You cannot copy and paste photos either.

\begin{sphinxadmonition}{warning}{Warning:}
\sphinxAtStartPar
If you search for Hubspot persona on Google, you might find
\sphinxhref{https://knowledge.hubspot.com/contacts/create-and-edit-personas}{this tutorial}.
This is \sphinxstylestrong{VERY DIFFERENT} to the Make My Persona tool. Please don’t use this one.
\end{sphinxadmonition}


\section{PersonaGenerator}
\label{\detokenize{appendices/appendix_c/personas_guide:personagenerator}}
\sphinxAtStartPar
This tool is pretty straight\sphinxhyphen{}forward. After you visit \sphinxhref{https://personagenerator.com/}{this link}, you can start
creating your persona by filling in content.


\subsection{Save and Export}
\label{\detokenize{appendices/appendix_c/personas_guide:id1}}\begin{itemize}
\item {} 
\sphinxAtStartPar
You get two links: one is view\sphinxhyphen{}only, one allows editing. You can share the two links in your team to edit/view
your personas.

\item {} 
\sphinxAtStartPar
You can print the persona as a PDF.

\end{itemize}


\subsection{Any disadvantages?}
\label{\detokenize{appendices/appendix_c/personas_guide:id2}}
\sphinxAtStartPar
You \sphinxstylestrong{cannot} change the section title, add a new section or delete a section. In other words, the sections you can
have on your persona are fixed. Please consider this before you choose this tool.


\section{UXPRESSIA}
\label{\detokenize{appendices/appendix_c/personas_guide:uxpressia}}\begin{enumerate}
\sphinxsetlistlabels{\arabic}{enumi}{enumii}{}{.}%
\item {} 
\sphinxAtStartPar
To use this tool, create an account with UXPRESSIA for free. Visit
\sphinxhref{https://uxpressia.com/personas-online-tool}{this link}.

\item {} 
\sphinxAtStartPar
Firstly, you need to sign up.

\item {} 
\sphinxAtStartPar
After you have signed up successfully, you will arrive at your personal workspace. Create a project.
\sphinxincludegraphics{{UXPRESSIA1}.JPG}

\item {} 
\sphinxAtStartPar
In your project, you can create your persona. Click “ADD NEW”
\sphinxincludegraphics{{UXPRESSIA2}.JPG}

\item {} 
\sphinxAtStartPar
Select “PERSONAS”. You can start with a blank one, then edit it yourself.
\sphinxincludegraphics{{UXPRESSIA3}.JPG}

\item {} 
\sphinxAtStartPar
Uxpressia allows you to be very flexible with your sections \sphinxhyphen{} you can add/delete them, and there are different
types of sections to choose from (e.g. text, sliders).
\sphinxincludegraphics{{UXPRESSIA4}.JPG}

\end{enumerate}


\subsection{Any Disadvantages?}
\label{\detokenize{appendices/appendix_c/personas_guide:id3}}
\sphinxAtStartPar
The following restrictions are limitations of a free plan:
\begin{enumerate}
\sphinxsetlistlabels{\arabic}{enumi}{enumii}{}{.}%
\item {} 
\sphinxAtStartPar
You can only create one persona per project.

\item {} 
\sphinxAtStartPar
You can only create one project per account

\item {} 
\sphinxAtStartPar
You can only export it as a PNG.

\item {} 
\sphinxAtStartPar
To share your persona, you need to share your project first. The free plan only let you share your project with
1 person via email.

\end{enumerate}

\begin{sphinxadmonition}{tip}{Tip:}
\sphinxAtStartPar
You could form small teams of 2 within the team to work on a persona. For review, the smaller teams can
exchange their personas. You will need to create multiple accounts with UXPRESSIA.
\end{sphinxadmonition}


\section{Xtensio}
\label{\detokenize{appendices/appendix_c/personas_guide:xtensio}}
\sphinxAtStartPar
Visit \sphinxhref{https://xtensio.com/user-persona-template/}{this link} to start exploring.


\subsection{Any Disadvantages?}
\label{\detokenize{appendices/appendix_c/personas_guide:id4}}
\sphinxAtStartPar
The following restrictions are limitations of a free plan:
\begin{enumerate}
\sphinxsetlistlabels{\arabic}{enumi}{enumii}{}{.}%
\item {} 
\sphinxAtStartPar
You cannot share the work with anyone.

\item {} 
\sphinxAtStartPar
You cannot download it as a PNG or PDF.

\item {} 
\sphinxAtStartPar
Your work expires after a couple of hours.
\sphinxincludegraphics{{Xtensio1}.JPG}

\end{enumerate}

\sphinxAtStartPar
We do not recommend the free version of this tool due to the above limitations.  If your team has explored other
tools and decided that you want to use this one, check out their
\sphinxhref{https://xtensio.com/pricing/?\_gl=1*rkelh8*\_ga*OTU0NjY3MjY3LjE2NDIzOTQ0ODE.*\_ga\_EFSR9CLTP4*MTY0MjM5NDQ4MC4xLjEuMTY0MjM5NTU2Mi41Nw..}{pricing}.


\chapter{Appendix D: Miro Tutorial}
\label{\detokenize{appendices/appendix_d/miro_guide:appendix-d-miro-tutorial}}\label{\detokenize{appendices/appendix_d/miro_guide::doc}}
\sphinxAtStartPar
Miro is an online collaborative whiteboard platform that can be useful to you in SWEN90009. In this tutorial, we
will first help you set up an account. Then, we discuss how Miro can be useful for some tasks in SWEN90009.


\section{Set Up}
\label{\detokenize{appendices/appendix_d/miro_guide:set-up}}\begin{enumerate}
\sphinxsetlistlabels{\arabic}{enumi}{enumii}{}{.}%
\item {} 
\sphinxAtStartPar
Sign up for free on \sphinxhref{https://miro.com/}{Miro website}.

\item {} 
\sphinxAtStartPar
Once you have signed up, Miro will let you set up your team. (Note: If you use your university student
email address to sign up, you might notice that there are some teams that you can join in your organisation
already \sphinxhyphen{} but that’s not what we need. In this subject, you will need to create your own project team. )
\sphinxincludegraphics{{Miro1}.JPG}

\item {} 
\sphinxAtStartPar
Invite your team members.\sphinxincludegraphics{{Miro2}.JPG}

\end{enumerate}


\section{Tips Before you Start}
\label{\detokenize{appendices/appendix_d/miro_guide:tips-before-you-start}}

\subsection{Number of Editable Boards}
\label{\detokenize{appendices/appendix_d/miro_guide:number-of-editable-boards}}\begin{itemize}
\item {} 
\sphinxAtStartPar
For a free account, your team only has 3 editable boards. Once you create more than 3 boards, only the 3 most
recently created ones are editable.

\item {} 
\sphinxAtStartPar
How to make sure that you do not run out of boards? You can draw multiple diagrams on one board, and put each
diagram inside a {\hyperref[\detokenize{appendices/appendix_d/miro_guide:frame}]{\emph{frame}}}.

\end{itemize}


\subsection{Template}
\label{\detokenize{appendices/appendix_d/miro_guide:template}}
\sphinxAtStartPar
Miro has a library of templates to help you get started with your task. There are two main ways to explore and
add templates:
\begin{enumerate}
\sphinxsetlistlabels{\arabic}{enumi}{enumii}{}{.}%
\item {} 
\sphinxAtStartPar
Option 1: On your Miro main page, you should see a list of templates suggestions. Click on a template to create a
new board with the chosen template, or select “Show all templates” to explore more templates options.
\sphinxincludegraphics{{Miro3}.JPG}

\item {} 
\sphinxAtStartPar
Option 2: Open a board you have already created, and add a new template in.
\sphinxincludegraphics{{Miro4}.JPG}

\end{enumerate}


\subsection{Frame}
\label{\detokenize{appendices/appendix_d/miro_guide:frame}}
\sphinxAtStartPar
Frame can help you organise your board, especially if you run out of boards and decide to put multiple diagrams in
one board. Also, if you want to export your board to a PDF, it is essential that you put content inside a frame
(each frame will be a single page on the PDF).

\sphinxAtStartPar
\sphinxstylestrong{How to add a frame:}
\begin{enumerate}
\sphinxsetlistlabels{\arabic}{enumi}{enumii}{}{.}%
\item {} 
\sphinxAtStartPar
On the tool bar, select the frame icon (marked in red).

\item {} 
\sphinxAtStartPar
Then, you can select the desired frame dimention. In this example, let us select “custom”.\sphinxincludegraphics{{Miro5}.JPG}

\item {} 
\sphinxAtStartPar
Ajust the size of the frame so that your artefact is completely covered in the frame, as shown in the image
below. You can give this frame a title, and change the background colour of your frame.
\sphinxincludegraphics{{Miro6}.JPG}

\end{enumerate}


\section{Use Miro}
\label{\detokenize{appendices/appendix_d/miro_guide:use-miro}}
\sphinxAtStartPar
Miro allows you and your team to interact and collaborate on this platform. We strongly encourage you to use that to
your advantage. {\hyperref[\detokenize{appendices/appendix_d/miro_guide:miro-for-brainstorming}]{\emph{Brainstorming}}} and
{\hyperref[\detokenize{appendices/appendix_d/miro_guide:miro-for-sprint-retrospective}]{\emph{Sprint Retrospective meeting}}} are two great examples that you can use Miro to have
interactive team meetings.

\sphinxAtStartPar
Miro has a lot of templates at your disposal.  In addition to brainstorming and retrospective meeting, check out the
one for {\hyperref[\detokenize{appendices/appendix_d/miro_guide:miro-for-user-story-map}]{\emph{User Story Map}}}.


\subsection{Miro for Brainstorming}
\label{\detokenize{appendices/appendix_d/miro_guide:miro-for-brainstorming}}
\sphinxAtStartPar
Let’s walk through an example. Let’s say you are designing an application that will help students decide which
subject to take in the future.
\begin{enumerate}
\sphinxsetlistlabels{\arabic}{enumi}{enumii}{}{.}%
\item {} 
\sphinxAtStartPar
Write down the question that needs to be solved on your board. You can add it as text.

\item {} 
\sphinxAtStartPar
Add the “Brainwriting” template to your board. Check out the {\hyperref[\detokenize{appendices/appendix_d/miro_guide:template}]{\emph{Template}}} section on how to add a
template on a board.
\sphinxincludegraphics{{Miro7}.JPG}

\item {} 
\sphinxAtStartPar
You will find that the sticky notes are already created for you, ready to be filled in. Here are some tips
for you:
\begin{itemize}
\item {} 
\sphinxAtStartPar
You can use tag to help you organise your ideas into categories.

\item {} 
\sphinxAtStartPar
You can add reaction to ideas
\sphinxincludegraphics{{Miro8}.JPG}

\end{itemize}

\item {} 
\sphinxAtStartPar
You can also add comments on the ideas
\begin{itemize}
\item {} 
\sphinxAtStartPar
Select the comment tool in the tool bar

\item {} 
\sphinxAtStartPar
Click on one of the sticky notes, and start adding comments
\sphinxincludegraphics{{Miro9}.JPG}

\end{itemize}

\end{enumerate}

\sphinxAtStartPar
You don’t have to start with any template. Feel free to start with an empty board and create sticky notes as you
run the brainstorm session.


\subsection{Miro for Sprint Retrospective}
\label{\detokenize{appendices/appendix_d/miro_guide:miro-for-sprint-retrospective}}
\sphinxAtStartPar
Miro has lots of templates for Retrospective meeting.  Search for “retrospective” in their template pool and
select the one that suits your team the best.  Retrospective Start/Stop/Continue is a template that we would
recommend.


\subsection{Miro for User Story Map}
\label{\detokenize{appendices/appendix_d/miro_guide:miro-for-user-story-map}}\begin{enumerate}
\sphinxsetlistlabels{\arabic}{enumi}{enumii}{}{.}%
\item {} 
\sphinxAtStartPar
Miro has two templates for User Story Map. Feel free to choose one and add it to your board. In this example,
we will select the basic one.
\sphinxincludegraphics{{Miro10}.JPG}

\item {} 
\sphinxAtStartPar
Fill in the template and {\hyperref[\detokenize{appendices/appendix_d/miro_guide:export}]{\emph{export}}} your diagram.
\sphinxincludegraphics{{Miro11}.JPG}

\end{enumerate}


\section{Export}
\label{\detokenize{appendices/appendix_d/miro_guide:export}}
\sphinxAtStartPar
There are two main ways to export your diagrams: export as an image or PDF.
\begin{enumerate}
\sphinxsetlistlabels{\arabic}{enumi}{enumii}{}{.}%
\item {} 
\sphinxAtStartPar
Click the Export option at the top of your board.\sphinxincludegraphics{{Miro12}.JPG}

\item {} 
\sphinxAtStartPar
Either export as an image or PDF.  If you want to export it as a PDF, make sure you add a {\hyperref[\detokenize{appendices/appendix_d/miro_guide:frame}]{\emph{frame}}}.\sphinxincludegraphics{{Miro13}.JPG}

\end{enumerate}

\begin{sphinxadmonition}{note}{Extra Resources}

\sphinxAtStartPar
There are so much more you can explore with Miro!  For more tutorials and help,
feel free to check out their \sphinxhref{https://help.miro.com/hc/en-us}{website}.
\end{sphinxadmonition}


\chapter{Appendix E: Marvel Tutorial}
\label{\detokenize{appendices/appendix_e/marvel_guide:appendix-e-marvel-tutorial}}\label{\detokenize{appendices/appendix_e/marvel_guide::doc}}
\sphinxAtStartPar
This tutorial shows you how you can use Marvel to help you create your paper prototype.


\section{Before You Start Prototyping}
\label{\detokenize{appendices/appendix_e/marvel_guide:before-you-start-prototyping}}
\sphinxAtStartPar
Before you start prototyping, let us remind ourselves \sphinxhyphen{} why are we creating paper prototypes?  What’s the purpose
again?

\sphinxAtStartPar
The purpose of creating paper prototypes is to design the user interfaces of the system\sphinxhyphen{}to\sphinxhyphen{}be so that you can run
usability test with end\sphinxhyphen{}users and clients to make sure it is fit for purpose. If you discover any problem during
usability testing, you can go back and modify your design.  By the end of the process, you have a validated
prototype before you start coding.  Improving design on your prototype is probably going to be much less expansive
than on your code.

\sphinxAtStartPar
Now that you are clear on the purpose behind the task, we would like to provide some advice to help you:


\subsection{No.1: Completeness of Scenarios}
\label{\detokenize{appendices/appendix_e/marvel_guide:no-1-completeness-of-scenarios}}
\sphinxAtStartPar
During your usability test, you will have tasks for your users/clients to perform. It’s really important to make
sure that these scenarios in your paper prototype are complete. What does that mean? Let me give you an example.

\sphinxAtStartPar
Imagine that you have a scenario where a user needs to book a hotel room.  On the interface, you have an option to
confirm or cancel. You made sure that the “Confirm” button is clickable, and you prepared the transition to the
next page.  But the “Cancel” button is non\sphinxhyphen{}clickable. During the usability test, your client clicks “Cancel” and
nothing happens.

\sphinxAtStartPar
The example above is an \sphinxstylestrong{incomplete} scenario. \sphinxstylestrong{It’s better to have 4 \sphinxhyphen{} 5 complete scenarios than creating 100
pages but a lot of buttons are non\sphinxhyphen{}clickable}.  Remember \sphinxhyphen{} you want to run usability test and observe your users
interacting with the system to improve your design. Half\sphinxhyphen{}completed scenarios won’t help you with that \sphinxhyphen{} it will even
create confusions for users/clients. If you are designing a large system and there are a LOT of user stories to
cover, we advise you to:
\begin{enumerate}
\sphinxsetlistlabels{\arabic}{enumi}{enumii}{}{.}%
\item {} 
\sphinxAtStartPar
Ask yourself: what are the most important things that users must be able to accomplish on the application?
What user stories MUST be tested?

\item {} 
\sphinxAtStartPar
Select 4 \sphinxhyphen{} 5 user stories or scenarios from your list, and start prototyping. Make sure they are complete.

\end{enumerate}


\subsection{No.2: Don’t Give the Answer Away}
\label{\detokenize{appendices/appendix_e/marvel_guide:no-2-don-t-give-the-answer-away}}
\sphinxAtStartPar
Let’s use our booking hotel room example again. Make sure you don’t give away clues or describe the steps to your
users. Don’t say “First you would login, then you would click the Find Destination button, then you would browse
all the hotels available, then…” the instructions that you give will prevent you from discovering design flaws.
Instead, set up a little scenario for your users, provide them with context and observe how they complete the
activity. For example, a better task would be:

\sphinxAtStartPar
\sphinxstyleemphasis{You are planning a trip with your family to Sydney from 24th December \sphinxhyphen{} 2nd January. You want to book a hotel room.
Visit our site and see what’s on offer.}


\subsection{No.3: Remember This Is a Low\sphinxhyphen{}Fidelity Prototype}
\label{\detokenize{appendices/appendix_e/marvel_guide:no-3-remember-this-is-a-low-fidelity-prototype}}
\sphinxAtStartPar
Low\sphinxhyphen{}fidelity prototype shows only major navigation and content elements. You don’t need to include details such as
the colour scheme, images, styling, or meaningful content.


\subsection{Want Some More Resources?}
\label{\detokenize{appendices/appendix_e/marvel_guide:want-some-more-resources}}
\sphinxAtStartPar
Check out some tutorials and resources for running usability tests in the {\hyperref[\detokenize{appendices/appendix_e/marvel_guide:more-resources}]{\emph{More Resources}}} section.


\section{Tips For Marvel}
\label{\detokenize{appendices/appendix_e/marvel_guide:tips-for-marvel}}
\sphinxAtStartPar
Here are some tips for using Marvel to create your prototype:
\begin{enumerate}
\sphinxsetlistlabels{\arabic}{enumi}{enumii}{}{.}%
\item {} 
\sphinxAtStartPar
Make sure that you hide hotspot hints
(\sphinxhref{https://help.marvelapp.com/hc/en-us/articles/360002746798-How-to-hide-hotspot-hints-when-playing-prototypes}{instructions})
when testing your prototype with end\sphinxhyphen{}users/clients. Remember, one of the goals of paper prototype is to test the
usability of your design. If you don’t hide hotspot hints, end\sphinxhyphen{}users/clients will be given clues to know where to
click.

\item {} 
\sphinxAtStartPar
Even with 4 \sphinxhyphen{} 5 scenarios, you might end up creating many of screens for your paper prototype. You can organise
your screens into \sphinxstylestrong{sections} on Marvel. Checkout this series of short
\sphinxhref{https://help.marvelapp.com/hc/en-us/sections/360001413858-Organising-designs-with-sections}{tutorials}
on sections.

\end{enumerate}


\section{Initial Set Up}
\label{\detokenize{appendices/appendix_e/marvel_guide:initial-set-up}}\begin{enumerate}
\sphinxsetlistlabels{\arabic}{enumi}{enumii}{}{.}%
\item {} 
\sphinxAtStartPar
Follow \sphinxhref{https://marvelapp.com/}{this link} to create an Marvel account.

\item {} 
\sphinxAtStartPar
During your sign up process, Marvel might ask you to create your first Marvel Whiteboard. You can choose “Skip for now” to ignore this option, because what you will be needing is \sphinxstylestrong{NOT} a whiteboard.\sphinxincludegraphics{{Marvel1}.JPG}

\item {} 
\sphinxAtStartPar
If you want to invite your team members to your workspace, keep in mind that for a free account, you can have 6 people in TOTAL (including yourself!) to \sphinxstylestrong{edit} the workspace projects.

\item {} 
\sphinxAtStartPar
Using a free account, you only get ONE editable project. Keep that in mind before you create a second project.

\end{enumerate}


\section{Create Your First Project}
\label{\detokenize{appendices/appendix_e/marvel_guide:create-your-first-project}}\begin{enumerate}
\sphinxsetlistlabels{\arabic}{enumi}{enumii}{}{.}%
\item {} 
\sphinxAtStartPar
After you have signed up, you can “Create project”.\sphinxincludegraphics{{Marvel2}.JPG}

\item {} 
\sphinxAtStartPar
There will be options to choose from \sphinxhyphen{} select the “Prototype” option.\sphinxincludegraphics{{Marvel3}.JPG}

\item {} 
\sphinxAtStartPar
Then, enter the project name.

\item {} 
\sphinxAtStartPar
Do you know if your client wants a web\sphinxhyphen{}application, or a mobile app? If it is a web\sphinxhyphen{}application, will the users mostly access this web\sphinxhyphen{}app via their phone, or laptop/computer? Select a device accordingly. In this tutorial, we will go with iPhone 12. \sphinxincludegraphics{{Marvel4}.JPG}

\item {} 
\sphinxAtStartPar
Create project.

\end{enumerate}


\section{Prototyping}
\label{\detokenize{appendices/appendix_e/marvel_guide:prototyping}}\begin{itemize}
\item {} 
\sphinxAtStartPar
Once you have created your project, you can start prototyping! There are two stages to create your paper prototype:
\begin{enumerate}
\sphinxsetlistlabels{\arabic}{enumi}{enumii}{}{.}%
\item {} 
\sphinxAtStartPar
Design your screens \sphinxhyphen{} what are they going to look like?

\item {} 
\sphinxAtStartPar
Design the interactions and transitions between your screens.

\end{enumerate}

\item {} 
\sphinxAtStartPar
Marvel offers three main ways to design your screens:
\begin{enumerate}
\sphinxsetlistlabels{\arabic}{enumi}{enumii}{}{.}%
\item {} 
\sphinxAtStartPar
Create your own design on Marvel.

\item {} 
\sphinxAtStartPar
If you have got some design already, you can upload your design in Marvel.

\item {} 
\sphinxAtStartPar
Use the template that Marvel provides.
\sphinxincludegraphics{{Marvel5}.JPG}

\end{enumerate}

\end{itemize}


\section{Prototyping \sphinxhyphen{} An Example}
\label{\detokenize{appendices/appendix_e/marvel_guide:prototyping-an-example}}
\sphinxAtStartPar
Let’s walk through an example.  Let’s say you are designing an application for universities that will help university
students decide which subjects to take in the future. Your clients have told you that they want the application to
be a mobile app.

\begin{sphinxadmonition}{note}{Reminder}

\sphinxAtStartPar
Your paper prototype is an early draft of your design \sphinxhyphen{} it should be rapidly designed, simulated and tested with
users. Please be careful with the level of details you put into it.  It needs to show major navigation and major
content elements. It does \sphinxstylestrong{NOT} need to show colours, images or meaningful contents.
\end{sphinxadmonition}
\begin{enumerate}
\sphinxsetlistlabels{\arabic}{enumi}{enumii}{}{.}%
\item {} 
\sphinxAtStartPar
Create your first project according to the {\hyperref[\detokenize{appendices/appendix_e/marvel_guide:create-your-first-project}]{\emph{create your first project}}} section.

\item {} 
\sphinxAtStartPar
Let’s design the first page \sphinxhyphen{} Home page.
\begin{itemize}
\item {} 
\sphinxAtStartPar
Click on “Start designing” \sphinxincludegraphics{{Marvel6}.JPG}

\item {} 
\sphinxAtStartPar
Firstly, our Home page can show some basic information about the user who has logged in. Let’s create a profile
image icon. Select the icon options on the toolbar.\sphinxincludegraphics{{Marvel7}.JPG}

\item {} 
\sphinxAtStartPar
Let’s pick the 2nd one, and add it to our design.\sphinxincludegraphics{{Marvel8}.JPG}

\item {} 
\sphinxAtStartPar
Secondly, we can add some basic information about the user. Select the text option, and add text in our design.\sphinxincludegraphics{{Marvel9}.JPG}

\item {} 
\sphinxAtStartPar
For now, our page looks like:\sphinxincludegraphics{{Marvel10}.JPG}

\item {} 
\sphinxAtStartPar
Next, we decide that some subjects should be listed.  They should be categorised into “Compulsory” (student
must complete this subject as part of the course) and “Optional”. We can add some rectangles to represent subjects:\sphinxincludegraphics{{Marvel11}.JPG}

\item {} 
\sphinxAtStartPar
Our page looks like this now:\sphinxincludegraphics{{Marvel12}.JPG}

\item {} 
\sphinxAtStartPar
We are done with the design.  Let’s close this page.\sphinxincludegraphics{{Marvel13}.JPG}

\end{itemize}

\item {} 
\sphinxAtStartPar
Let’s design the next page \sphinxhyphen{} a subject’s page.
\begin{itemize}
\item {} 
\sphinxAtStartPar
Let’s create a new image.\sphinxincludegraphics{{Marvel14}.JPG}

\item {} 
\sphinxAtStartPar
According to our clients’ requirements, a subject’s page should have the subject’s name, a description, area of
interests it relates to, advice from teaching staffs and past students, and a button to go back to the home page.
The outcome might look like:\sphinxincludegraphics{{Marvel15}.JPG}

\end{itemize}

\item {} 
\sphinxAtStartPar
Let’s say we are done with our design our pages (for now).  Let’s design the interactions between the two pages
we created.
\begin{itemize}
\item {} 
\sphinxAtStartPar
Hover your mouse over the Home page, click “Prototype” to add interactions.
\sphinxincludegraphics{{Marvel16}.png}

\item {} 
\sphinxAtStartPar
Click and drag to draw a hotspot (hotspot is an interactive area to enable users move between screens)
\sphinxincludegraphics{{Marvel17}.png}

\item {} 
\sphinxAtStartPar
Next, select the next screen you want to transition to. \sphinxincludegraphics{{Marvel18}.JPG}

\item {} 
\sphinxAtStartPar
Next, select the transition type. \sphinxincludegraphics{{Marvel19}.JPG}

\item {} 
\sphinxAtStartPar
Next, select the action that triggers the transition (i.e. is it a click, tap, or?) \sphinxincludegraphics{{Marvel20}.JPG}

\end{itemize}

\item {} 
\sphinxAtStartPar
Let’s say we are done with designing the interactions between screens. Select the “Play” button in the top right
corner to watch the transition you just created.
\sphinxincludegraphics{{Marvel21}.JPG}

\item {} 
\sphinxAtStartPar
To share paper prototype with your client so that you can run usability test with them, click “Share”, copy the
link, and send it with them. \sphinxincludegraphics{{Marvel22}.JPG}

\end{enumerate}

\sphinxAtStartPar
We hope that this simple example has helped you getting started with Marvel.

\begin{sphinxadmonition}{note}{Extra Resources}

\sphinxAtStartPar
For Usability Test
\begin{itemize}
\item {} 
\sphinxAtStartPar
This \sphinxhref{https://www.nngroup.com/articles/task-scenarios-usability-testing/}{tutorial} gives you more guidance and
advice on how you should run your usability tests.

\item {} 
\sphinxAtStartPar
This \sphinxhref{https://www.nngroup.com/articles/why-you-only-need-to-test-with-5-users/}{article} explains the number of
users to run usability tests with.

\end{itemize}

\sphinxAtStartPar
For Marvel
\begin{itemize}
\item {} 
\sphinxAtStartPar
Marvel has a good \sphinxhref{https://help.marvelapp.com/hc/en-us/articles/360002536038}{tutorial} to help you get started
with designing your screens and the interactions between screens \sphinxhyphen{} make sure to check it out!

\item {} 
\sphinxAtStartPar
To find more resources for prototyping on Marvel, visits
\sphinxhref{https://help.marvelapp.com/hc/en-us/categories/360000779958-Prototyping}{this link}.

\end{itemize}
\end{sphinxadmonition}


\chapter{Appendix E: Figma Tutorial}
\label{\detokenize{appendices/appendix_e/figma_guide:appendix-e-figma-tutorial}}\label{\detokenize{appendices/appendix_e/figma_guide::doc}}
\sphinxAtStartPar
This tutorial shows you how you can use Figma to help you create your digital prototype.


\section{Before you Start Prototyping}
\label{\detokenize{appendices/appendix_e/figma_guide:before-you-start-prototyping}}
\sphinxAtStartPar
You have completed your paper prototype and collected feedback from your clients.  Now you can use the feedback you
receive to create digital (high\sphinxhyphen{}fidelity) prototype that is much closer to what the final version of system\sphinxhyphen{}to\sphinxhyphen{}be
would look like.

\sphinxAtStartPar
Similar to paper (low\sphinxhyphen{}fidelity) prototype, you will use your digital prototype to run usability tests with
end\sphinxhyphen{}users/clients, collect more feedback, and refine your design. If you discover any problem during usability
testing, you can go back and modify your design.  By the end of the process, you have a validated prototype
before you start coding.

\sphinxAtStartPar
With that in mind, we want to remind you of some important key points:


\subsection{No.1: Digital Prototype Is High\sphinxhyphen{}Fidelity}
\label{\detokenize{appendices/appendix_e/figma_guide:no-1-digital-prototype-is-high-fidelity}}
\sphinxAtStartPar
Compared to low\sphinxhyphen{}fidelity prototype, high\sphinxhyphen{}fidelity prototype is much closer to the final version of your
system\sphinxhyphen{}to\sphinxhyphen{}be. You need to be a lot more specific, introducing colour scheme, styling, images and meaningful,
realistic content into it. There are not only major navigation elements, but also the more detailed ones in digital
prototype. It is what the real product would look like \sphinxhyphen{} except there is no code.


\subsection{No.2: Completeness of Scenarios}
\label{\detokenize{appendices/appendix_e/figma_guide:no-2-completeness-of-scenarios}}
\sphinxAtStartPar
During your usability test, you will have tasks for your users/clients to perform. It’s really important to make
sure that these scenarios in your digital prototype are complete. What does that mean? Let me give you an example.

\sphinxAtStartPar
Imagine that you have a scenario where a user needs to book a hotel room.  On the interface, you have an option to
confirm or cancel. You made sure that the “Confirm” button is clickable, and you prepared the transition to the
next page.  But the “Cancel” button is non\sphinxhyphen{}clickable. During the usability test, your client clicks “Cancel”
and nothing happens.

\sphinxAtStartPar
The example above is an \sphinxstylestrong{incomplete} scenario. \sphinxstylestrong{It’s better to have 4 \sphinxhyphen{} 5 complete scenarios than creating 100
pages but a lot of buttons are non\sphinxhyphen{}clickable}.  Remember \sphinxhyphen{} you want to run usability test and observe your users
interacting with the system to improve your design. Half\sphinxhyphen{}completed scenarios won’t help you with that \sphinxhyphen{} it will even
create confusions for users/clients. If you are designing a large system and there are a LOT of scenarios or user
stories to cover, we advise you to:
\begin{enumerate}
\sphinxsetlistlabels{\arabic}{enumi}{enumii}{}{.}%
\item {} 
\sphinxAtStartPar
Ask yourself: what are the most important things that users must be able to accomplish on the application?
What user stories MUST be tested?

\item {} 
\sphinxAtStartPar
Select 4 \sphinxhyphen{} 5 user stories or scenarios from your list, and start prototyping. Make sure they are complete.

\end{enumerate}


\subsection{No.3: Don’t Give the Answer Away}
\label{\detokenize{appendices/appendix_e/figma_guide:no-3-don-t-give-the-answer-away}}
\sphinxAtStartPar
Let’s use our booking hotel room example again. Make sure you don’t give away clues or describe the steps to your
users. Don’t say “First you would log in, then you would click the Find Destination button, then you would browse
all the hotels available, then…” the instructions that you give will prevent you from discovering design flaws.
Instead, set up a little scenario for your users, provide them with context and observe how they complete the
activity. For example, a better task would be:

\sphinxAtStartPar
\sphinxstyleemphasis{You are planning a trip with your family to Sydney from 24th December \sphinxhyphen{} 2nd January. You want to book a hotel room.
Visit our site and see what’s on offer.}

\begin{sphinxadmonition}{note}{Extra Resources}

\sphinxAtStartPar
If you haven’t checked these resources out, we highly recommend you have a look:
\begin{itemize}
\item {} 
\sphinxAtStartPar
This \sphinxhref{https://www.nngroup.com/articles/task-scenarios-usability-testing/}{tutorial} gives you more guidance
and advice on how you should run your usability tests.

\item {} 
\sphinxAtStartPar
This \sphinxhref{https://www.nngroup.com/articles/why-you-only-need-to-test-with-5-users/}{article} explains the number
of users to run usability tests with.

\end{itemize}

\sphinxAtStartPar
Figma is free for students. Check out \sphinxhref{https://www.figma.com/education/}{this website} to verify your education
status.
\begin{enumerate}
\sphinxsetlistlabels{\arabic}{enumi}{enumii}{}{.}%
\item {} 
\sphinxAtStartPar
Figma have some written \sphinxhref{https://help.figma.com/hc/en-us/sections/4403936156695-Build-prototypes}{tutorials} on how to build prototypes as well.

\item {} 
\sphinxAtStartPar
Figma have a series of videos tutorials for beginners, covering design creation and prototypes building. The first tutorial of the series is \sphinxhref{https://www.youtube.com/watch?v=dXQ7IHkTiMM}{here}.

\end{enumerate}
\end{sphinxadmonition}


\part{Frequently Asked Questions (FAQs)}







\renewcommand{\indexname}{Index}
\printindex
\end{document}